
\section{THE MODEL}

As explained above, the BVM is deceptively simple. Each node in the graph is
initially assigned an opinion (say, 0 or 1) and updates it periodically by
copying the opinion of one of its graph neighbors. It has long been known that
such a system will reach consensus (uniformity of one opinion or the other)
under a wide variety of conditions (see, \textit{e.g.},
\cite{sood_voter_2005}). The probability that 0 (as opposed to 1) becomes the
dominant opinion as a function of the initial opinion distribution is known,
as is the expected number of iterations required to reach consensus for
various degree distributions of the graph.

Implementing this model as an agent-based simulation is straightforward. Yet
in reproducing these classical results en route to other work, we discovered
at least two subtle implementation choices that at first glance would appear
unimportant, and yet which impact the convergence time in striking fashion. We
present these not so much as important in their own right, but as exemplars of
a more general problem: implementation choices that a modeler takes for
granted may turn out to be critical to the behavior claimed for that model.

\subsection{Simulation variant 1}

\subsection{Simulation variant 2}

\subsection{Analysis Methodology}






