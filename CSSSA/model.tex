
\label{sec:model}

We present the model using an abbreviated version of the ODD protocol
\cite{polhill_using_2008}.

\subsection{Purpose}

This abstract agent-based model simulates an evolving social network of agents
that each possess an ``ideology" attribute representing some opinion. As it
evolves, nodes are randomly chosen to influence their neighbors by propagating
their ideologies to them. The ``polarization" of the network, measured as the
tendency of nodes to be connected to others with the same ideology, will
change as this process takes place. 

The purpose of the model is to investigate the effects on polarization of two
parameters that affect how the initial graph is constructed. One,
``\textbf{accessibility}," controls the set of possible neighbors that a node
chooses from each time it adds a friendship. When accessibility is low,
neighbors will more often be chosen from a small (random) pool that is
initially available to a node. When it is high, neighbors will more often be
chosen from the entire network at large. Accessibility thus models the extent
to which a society's citizens form local friendships (\textit{e.g.},
geographical neighbors) versus global ones (\textit{e.g.}, over the Internet).

The second parameter, ``\textbf{homophily}," controls the strength of each
node's preference to attach to other nodes with the same ideology. When it is
high, nodes will almost always form connections to nodes that have the same
ideology, if possible. This is true regardless of whether they are formed
locally or globally. When low, nodes prefer to attach to nodes with
\textit{different} ideologies, and when medium, they are indifferent.

With this model, we hope to gain general insight into how societies that
provide different means of communication and discovery, and societies that
encourage different levels of tolerance for opposing opinions, may differ in
the prevalence of ``echo chambers" whereby dissenting views are rarely heard.

\subsection{Entities, State Variables and Scales}

The entities in the model are Agents, and have the following attributes:

\begin{description}
\itemsep.1em
\item[ID] A unique ID number.

\item[Ideology] One of a discrete set of possible ideologies, represented as
integers $0, 1, \dots I$. The ideology of an Agent will change during the
simulation, possibly many times, as it is influenced by its neighbors.

\item[Friendships] A set of references to other Agent entities with whom this
Agent has a social connection. The entire set of Agents and their Friendships
form an \textit{undirected} graph: if Agent $X$ is friends with Agent $Y$,
Agent $Y$ is also friends with Agent $X$.  Once the initial graph is built,
each Agent's set of Friendships is fixed over the lifetime of the model
(unless the \textbf{DynamicRebalancing} Policy is enabled; see below.)

\item[LocalAssociates] A set of references to other Agent entities with whom
this Agent \textit{may} form a friendship when it chooses locally (see below).
Conceptually, these represent the (small, relative to the whole population)
group of other people to whom an agent is geographically proximate. Some of
these may actually become the agent's friends; perhaps many or all of them if
the society has low accessibility. Importantly, once the initial graph is
constructed, the LocalAssociates attribute is discarded and no longer used in
the simulation.

\end{description}



\subsection{Process Overview and Scheduling}

Once the initial graph is built (see ``Initialization," below), we carry out
the standard Binary Voter Model process on it, using the \textbf{selection
with replacement} and \textbf{neighbor influences node} variants (see
\cite{davies_computational_2016}.) At each of $T$ iterations:

\begin{enumerate}
\itemsep.1em
\item An agent $X$ is chosen at random from the entire graph.
\item One of its friends $Y$ is chosen at random from its set of Friendships.
(If the agent has no Friendships, skip this iteration.)
\item If $Y$'s current ideology is different than $X$'s, copy $Y$'s to $X$'s.
\item If the \textbf{DynamicRebalancing} Policy is enabled, and its rate $R$
is such that it should take place now, carry out the
\textbf{DynamicallyRebalance} on agent $X$ (see below).
\item Compute and store the graph's nominal assortativity with respect to
Ideology (using the \texttt{igraph} package's \texttt{assortativity\_nominal}
function\cite{csardi_igraph_2006}).
\end{enumerate}


\subsection{Initialization}
\label{sec:initialization}

When the simulation begins, generate an undirected Erdos-Renyi random
graph\cite{erdos_evolution_1960} with $N$ nodes and probability of edge
connection $p$. Call this the \textbf{LocalAssociatesGraph}. Each node
represents an Agent, and the edges of this graph constitute its
\textbf{LocalAssociates}. With uniform probability, assign each node an
initial Ideology from the set of $I$ ideologies.

Then, generate the \textbf{Friendships} graph as follows:

\begin{enumerate}
\item Create an empty \textbf{Friendships} graph with $N$ nodes (agents).
\item For each node $X$ (in ID order), generate $a_X$ friendship connections
to other nodes, where $a_X$ is the degree of node $X$ in the
\textbf{LocalAssociatesGraph}. Choose the node for each connection as follows:
    \begin{itemize}
    \item With probability $A$ (the accessibility parameter), select a node
(without replacement) from $X$'s \textbf{LocalAssociates}, weighted by $H$
(the homophily parameter, see below).
    \item With probability $1-A$, select a node (without replacement) from all
nodes, weighted by $H$ (the homophily parameter, see below).
    \end{itemize}
    In either case, if there are no nodes available with whom $X$ is not
already friends, skip the step.
\end{enumerate}

``Weighted by $H$" means that candidate nodes whose ideologies are the same as
node $X$'s are assigned an (unnormalized) probability of $H$ to be selected,
and candidate nodes of different ideologies are assigned $1-H$. All
probabilities are then normalized to sum to 1, and a node is chosen.\footnote{
Therefore, if there are $c_X$ candidate nodes for $X$ to connect to, $s_X$ of
whose ideologies match $X$, the probability that each same-ideology node will
be chosen is $\alpha H$ and the probability for each different-ideology node
is $\alpha (1-H)$, where $\alpha$ is the normalizing constant $\frac{1}{s_X H
+ (c_X-s_X)(1-H)}$.}



\subsection{Submodels}

\textit{\textbf{DynamicRebalancing}.} If this optional Policy is enabled, an
``rebalance rate" parameter $R$ ($0 \leq R \leq 1$) controls how frequently
the \textbf{DynamicallyRebalance} submodel is executed. $R=1$ means it will
execute \textit{every} iteration of the main simulation loop (immediately
after copying the neighbor's ideology.) $R=0$ means it will never execute, and
any value in between means it will execute some fraction of the iterations.

When it executes for a node $X$, the \textbf{DynamicallyRebalance} process is
as follows:
\begin{enumerate}
\item Perform \textit{one} of the following:
\itemsep.1em
\begin{itemize}
\item With probability .5, connect a new node to $X$, chosen (without
replacement) from all other nodes, weighted by the homophily $H$ (exactly as
above).\footnote{In the unlikely event that $X$ is already connected to all
other nodes, skip this step.}
\item With probability .5, \textit{dis}connect a node from $X$ (breaking the
friendship), chosen from its current Friends, weighted by $1-H$. This means
that for $H>.5$, nodes with the same ideology as $X$ are \textit{less} likely
to be chosen than nodes with different ideologies.
\end{itemize}
\end{enumerate}



All of the numerical parameters are configurable. The simulation's default
values for them are $T=500, I=2, N=50, p=.06, A=.5, H=.7, R=0,$ and
\textbf{DynamicRebalancing} disabled.
