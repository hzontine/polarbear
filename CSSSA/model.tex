
\section{THE MODEL AND VARIANTS}

As explained above, the BVM is deceptively simple. Each node in the graph is
initially assigned an opinion (say, 0 or 1) and updates it periodically by
copying the opinion of one of its graph neighbors. It has long been known that
such a system will reach consensus (uniformity of one opinion or the other)
under a wide variety of conditions (see, \textit{e.g.},
\cite{sood_voter_2005}). The probability that 0 (as opposed to 1) becomes the
dominant opinion as a function of the initial opinion distribution is known,
as is the expected number of iterations required to reach consensus for
various degree distributions of the graph.

Implementing this model as an agent-based simulation is straightforward. Yet
in reproducing these classical results en route to other work, we discovered
at least two subtle implementation choices that at first glance would appear
unimportant, and yet which impact the convergence time in striking fashion. We
present these not so much as important in their own right, but as exemplars of
a more general problem: implementation choices that a modeler takes for
granted may turn out to be critical to the behavior claimed for that model.

\subsection{Simulation variant 1: choosing with or without replacement}

When we say ``each node periodically updates its opinion," what exactly does
that imply about the \textit{order} in which the nodes are selected for the
update process? There are at least two reasonable interpretations:

\begin{itemize}
\itemsep.1em

\item For each iteration of the simulation's main loop, a node is chosen
uniformly at random, \textit{with replacement}. In this scenario, if there
were five nodes in the graph, we might have the following sequence of choices
for opinion update: node 3, 5, 5, 5, 2, 3, 2, 5, 4, 2 \dots Notice that in
this realization, node 5 was selected 40\% of the time, node 1 was
not selected at all, and node 4 wasn't selected until near the end of the
sequence.

\item Treat \textit{each} of the nodes once (in random, shuffled order) before
treating them all again (in a different, shuffled order), \textit{etc.} Put
another way, chose the nodes uniformly at random \textit{without replacement}
until the store of nodes is exhausted; then replenish and repeat. In this
scenario, the sequence above would never happen; instead we might have
something like this: 3, 5, 2, 4, 1, 5, 4, 3, 2, 1, \dots In this way, every
node is guaranteed to be chosen once in $n=5$ iterations.

\end{itemize}

Clearly, based only on the system's English description, above, either of
these choices is consistent with the spirit of the model. An implementer might
choose either of them, either deliberately or (more probably) unconsciously.
They both pass the ``select nodes repeatedly in random fashion" test.

For the sake of clarity we adopt the terms ``\textbf{selection with
replacement}" and ``\textbf{selection without replacement}" as the
descriptions for these two alternatives.

Before examining the results, one question we might ask is: which of these two
variants is more reflective of the real-world phenomenon? Compelling arguments
can be made both ways. In favor of \textbf{selection without replacement} is
the observation that all human beings have 24 hours per day in which to live
and interact. If we want to simulate the dynamic behavior of a social system
over time, therefore, it is important to ensure that all agents act at a
fairly similar pace. After all, in the real world, there is no sequential loop
at all, agents are not successively ``chosen" to interact, and no agent is
``starved" for interaction as a consequence of a peculiar random number
sequence.

On the other hand, it is also true that in any social system, some agents will
be more active than others. Consider an online social network, such as
Facebook or Twitter. Some users post many messages per day, while others only
on rare occasions; and some \textit{read} many posts per day, whereas others
only glance at their customized news feeds once in a while. This heterogeneity
of usage would seem to favor the \textbf{selection with replacement} variant,
to do justice to the varying rates at which agents interact with one another.

\subsection{Simulation variant 2: direction of opinion propagation}

The model instructs us to (1) choose a node randomly, (2) choose one of its
graph neighbors, and then (3) have one of the nodes copy the opinion value of
the other. But which way does the influence go --- is the originally chosen
node the one whose opinion is updated, or is the neighbor's? At first glance,
this seems to be a symmetrical and arbitrary choice, and would not be expected
to impact the system's behavior. Indeed it does, however, and in a striking
way.

We define ``\textbf{node influences neighbor}" as the variant where the
originally selected node is the one whose opinion is propagated (to a random
one of its graph neighbors) and ``\textbf{neighbor influences node}" as the
alternative, where the copying goes from the node's neighbor back to it.

As to the question of which variant 2 choice is more reflective of reality, we
again find no definitive answers. In this case, our ambivalence stems from the
fact that the algorithm's influence procedure isn't itself a very good model
of what actually happens in real human interaction. Influence in a real human
conversation isn't (usually) unidirectional, with the ``winner" being
determined by considering which participant initiated the dialogue. It is a
complex process during which ideas are shared and defended, biases revealed, 
and opinions adjusted (or not) based on a myriad of factors. In a highly
abstract model such as this one, which does not aim to capture such
subtleties, how can we resolve this in the simplest and cleanest way? Clearly
by choosing from the two alternatives, above. Unfortunately, as we will see,
our choice here, too, made arbitrarily in the name of simplification,
noticeably changes the system's macro behavior.

\subsection{Analysis Methodology}






