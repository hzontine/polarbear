
The fact that $H_{1b}$ and $H_{2b}$ were refuted so soundly frankly astonished
us. But perhaps considering the literature in this area we should not be so
surprised. Different studies based on rather subtle extensions to the original
Axelrod model have come to very different conclusions as to whether more
accessibility will increase or decrease polarization. These differences appear
to depend on exactly how accessibility, polarization, homophily, and the
attribute in question are defined.

Clearly there is a delicate interplay here, at the heart of which remains an
unresolved question about what the end result of enhanced communication will
be for a society. Will the expanded freedom of selection lead to people
simply forming more homogeneous factions? Or will the greater exposure to
more ``remote'' parts of the society result in greater diversity of one's
social groups? Our model suggests the latter is more true, but the answer does
not appear to be simple. A careful study of the models presented in
section~\ref{sec:related} is called for in order to tease out which specific
differences are responsible for which effects. Then, the social psychology
question can be applied: how to synthesize aspects of these models to arrive
at a composite, more complex description of how human beings actually act?

As for $H_3$, the interaction between variables is interesting and
non-trivial. When agents exhibit a strong preference to form ties with their
own ``type'' and dissolve those with others, then changing friends often will
accelerate polarization, as expected. But if this homophily preference is
milder, not only will frequently changing friends fail to exacerbate
polarization, in fact it will dampen it.

Perhaps one interpretation is as follows. If a society is to be tolerant, its
citizens must each \textit{either} be open to friendships with people of
opposing views, \textit{or} be reluctant to dissolve ties and form new ones.
This somewhat tangled statement becomes intelligible if we consider that the
freedom to change friends freely is a dangerous tool in the hands of someone
with strong homophily. He is likely to use that tool to aggressively seek only
duplicates of himself. A person with low homophily, on the other hand, is
``safe'' to entrust that freedom to; she may change friends often, but that
will be a good thing, since she will form diverse ties.
