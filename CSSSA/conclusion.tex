
\section{CONCLUSION}

With this brief paper we hope simply to raise an alarm bell in the community
of agent-based social simulation. In the midst of the (admirable) efforts to
create abstract models reflective of reality, we have discovered that it is
easy for modelers to overlook the impact that presumably irrelevant
implementation choices may have on a simulation's behavior. The two examples
we raise in this article were ones we encountered in our own work, and which
surprised us a great deal. The silver lining was that this led us to redouble
our efforts to think through all aspects of our implementation a second time,
with an eye to unmasking choices we may have made without realizing that they
were important choices.

Only two remedies occur to us. The first is for ABM researchers to develop
this kind of scrutinizing mindset as a habit. Moving from the modeling phase
to the implementation phase of a project may not always just be a matter of
``cranking it out." It would be wise to adopt a cautious attitude and to
routinely second-guess every line of code: is the implementation decision I am
about to make a natural consequence of what the model demands? Or is this an
unwitting subjective choice which may nudge the system into giving a certain
outcome and not another, despite the fact that the model itself does not
prescribe either?

The second corrective measure is simply for the community to regularly
reproduce the work of peers. We hear much about ``reproducible research" in
the ABM community these days, but perhaps less about actually
\textit{reproducing} such research. If models are theoretically reproducible
but in practice rarely are, the dangers of unrealized assumptions lurking in
computational social science results will have few opportunities to be
exposed. For our part, we have resolved to more consistently attempt to
replicate the results of publications in our areas of interest in the future,
in order to learn more about the details of these models, and to help our
community produce even stronger, more transparent science.

\section*{ACKNOWLEDGEMENTS}

We are grateful to Dr.~David Marchette and to the UMW Summer Science Institute
for ideas, energy, and financial support.
