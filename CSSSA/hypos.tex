
\label{sec:hypos}

We form the following hypotheses about the above model's behavior.

\begin{description}

\item{\textbf{Hypothesis 1a ($H_{1a}$):}} For a given level of accessibility
$A$, the polarization $P$ \textit{of the initial graph} will increase with
homophily $H$.

\item{\textbf{Hypothesis 1b ($H_{1b}$):}} For a given $A$, $P$ will
\textit{continue} to increase with $H$ as the BVM process takes place on the
graph.

\item{\textbf{Hypothesis 2a ($H_{2a}$):}} For a given $H$>.5, the $P$ \textit{of
the initial graph} will increase with $A$.

\item{\textbf{Hypothesis 2b ($H_{2b}$):}} For a given $H$>.5, $P$ will
\textit{continue} to increase with $A$ as the BVM process takes place on the
graph.

\item{\textbf{Hypothesis 3 ($H_3$):}} If the \textbf{DynamicRebalancing}
policy is enabled, and $H$>.5, $P$ will increase more rapidly during the BVM
process, and more rapidly with increasing $R$. This will be unaffected by $A$.

\end{description}

We expect $H_{1a}$ to hold simply because as $H$ increases, the preference for
nodes to attach to others of the same ideology increases, which should
increase assortativity. Less obviously, $H_{2a}$ should hold because when $A$
increases nodes have greater freedom of choice in who they connect to. Low
values of $A$ mean that as the graph is built, nodes will often be ``forced''
to choose a friend from the limited number present in their LocalAssociates
set, and will thus more often have no choice but to make a friend of a
different ideology. Thus we expect more cross-ideology friendships when $A$ is
low. (Obviously this only holds when the homophily is greater than .5,
indicating a preference for same-ideology nodes.)

The rationale behind $H_{1b}$ and $H_{2b}$ is as follows. When the initial
graph is constructed so as to be more polarized, the BVM will have more raw
material to work with in order to make it further polarized. We expect
increasing returns as the structure of the network, already built to put
like-minded nodes largely together, causes the remaining vestiges of local
disagreement to be snuffed out.

Finally, we predict $H_3$ because dynamically rebalancing should have a
strictly positive impact on assortativity (when $H>.5$). New edges will be
added to the graph, and old edges dropped, with preference to
forming/maintaining friendships with likeminded nodes. This should accelerate
the degree to which the graph becomes polarized, and the rate of acceleration
should increase when we add/remove friendships more often. We have no
\textit{a priori} reason to suspect that the $A$ used to form the initial
graph will come into play.
