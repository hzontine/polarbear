
\section{INTRODUCTION}

In this work, we consider Opinion Dynamics models, which seek to reproduce
the phenomenon of individual agents forming opinions over time via mutual
influence. The field boasts a large literature, full of claims about the
behavior of various such systems, some supported with mathematical proofs,
others sustained by empirical evidence from simulation results.

When implementing any agent-based simulation, the designer faces choices that
seem inconsequential --- ``In which order should I update the state variables
here?" ``Should agents be treated in a consistent order, or should they be
shuffled each time?" ``Do I use a while loop or a for loop in this function?"
``When simultaneously inserting several events into the event queue, which one
do I enqueue first?" \textit{Etc.} Yet despite their apparent unimportance, in
some cases, system behavior may actually hinge on what choice is made. If this
goes undiscovered, there is a risk that broad claims made about a general
class of system may in fact be contingent on certain non-obvious
implementation concerns.

Among other things, this underscores the importance of the ABM community
taking the time to reproduce results from the literature. If several
researchers, starting from the same conceptual description of a system,
independently build implementations that produce the same macro-level
behavior, this increases confidence that the claims made about the system are
indeed robust. If not, this may expose the presence of latent
implementation-dependent assumptions that should actually be promoted to the
model description proper, rather than being omitted and hence left to an
implementer. It is then worth considering whether the augmented model, with
the new assumptions made explicit, is still a reasonable abstraction of the
real-world system being studied.

In this paper, we look at two such simulation variants of possibly the
simplest of all Opinion Dynamics models: the original Binary Voter Model
\cite{holley_ergodic_1975,clifford_model_1973}. We argue that in neither case
is one implementation choice obviously preferable to the other, and yet what
may seem a matter of indifference actually has a profound effect on the
runtime characteristics of the model -- in this case, the convergence time to
consensus.

