
\section{INTRODUCTION}

Opinion Dynamics models seek to reproduce the phenomenon of individual agents
forming opinions over time via mutually influencing one another. Various
classical models have been proposed, varying in 
the way they represent opinions --- 
    discrete \cite{follmer_random_1974,yildiz_discrete_2011} or
    continuous \cite{ghaderi_opinion_2012,weisbuch_interacting_2001},
    single \cite{weisbuch_dynamical_1999} or 
    multiple \cite{deffuant_mixing_2000,sirbu_opinion_2013},
    expressed or latent \cite{friedkin_social_1990};
how agents encounter each other ---
    (randomly from the whole population \cite{hegselmann_opinion_2002}, % cite DW too
    neighbors in a social network \cite{clifford_model_1973,holley_ergodic_1975},
    pairwise or in groups \cite{degroot_reaching_1974});
how influence takes place ---
    (copying another agent's opinion \cite{holley_ergodic_1975},
    averaging their opinion with one's own,  % cite what for this?
    ``disagreement" processes \cite{sirbu_opinion_2013});
and whether it takes place at all ---
    (bounded confidence models \cite{hegselmann_opinion_2002}, % cite DW too
    ``stubbornness" of some nodes \cite{yildiz_discrete_2011}).


