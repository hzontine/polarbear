
\subsection{Two sources of friendship}

Consider for a moment the friendships you have had through which meaningful
mutual influence has taken place. In broad terms, these relationships can be
thought of as coming from two different kinds of sources. In the first kind,
you did not originally encounter the acquaintances by specifically seeking
them out -- they rather made an appearance in your life due to circumstances.
When you were a child, there were other children down the street, and fellow
students in your 1st-grade classroom. When you were older, there were
dormitory hall-mates, new neighbors in new neighborhoods, and co-workers. Some
of these people who by happenstance wandered into your field of view, you
formed meaningful friendships with. Others you did not. But the key point that
distinguishes them from the second group is that you did not discover them by
seeking them out based any of their attributes. Instead, a minuscule set of
people (out of all the people in the world) simply fell into your lap. And
from that set, you became meaningful friends with some.

The second group consists of those you encountered \textit{because} you had
something in common, and you discovered them because you deliberately sought
that something. Even before the advent of cheap, electronic communication this
occurred: people attended churches and synagogues, knowing they would find
others with similar worldviews. They joined rotary clubs, parent-teacher
organizations, and political parties, seeking out those with similar
interests. In the Internet age, this is multiplied tenfold. Anyone with any
view or interest, no matter how esoteric, can find a chatroom, website, Google
group, subreddit, Twitter community, or other form of online clique devoted to
it. For the younger generation especially, this is an important group: recent
surveys suggest that a majority of American teens form meaningful friendships
online, and communicate with them daily, often never meeting the friend
physically.\cite{amanda_lenhart_teens_2015}

The key point is that some of your friends were ``chosen" from a very tiny
subset of the people in the world who you couldn't help but run into. You
might not have had anything in particular in common with them, other than
geographical proximity. The others, however, you drew from a very large pool
-- essentially the entire online world.\footnote{Note that once a relationship
has been established, the partners may never again actively consider the
source of the friendship: a link has simply been made, regardless of the
cause, and thereafter results in mutual influence.} And the tools of the
Internet give you breathtaking precision with which to find and select such
friends.

\subsection{``Accessibility"}

Admittedly, these two sources of friendship are idealized points along a
continuum. Any individual relationship may have been formed due to some blend
of the two. Nevertheless, it is a useful abstraction, and suggests the
existence of a key parameter in modeling social networks: the relative
strength of the two sources in leading to friendships. Put another way: on
average, what fraction of a person's friendships develop as a result of
geographical proximity and happenstance encounters, versus being due to more
distant relationships where parties sought each other out based on some shared
attribute? 

In the language of this paper, we use the term \textbf{local} to describe
friendships of the first sort, and \textbf{global} for the second. In our
model we are concerned with the relative likelihood of friendships developing
from each of these two sources, and we denote as the society's
\textbf{accessibility} $A$ ($0 \leq A \leq 1$) the fraction of relationships
that have \textit{global} origination. A society with an $A$ of 0 presents no
way for its citizens to discover anyone outside the local social circle they
happened to inherit. Persons in a society with $A=1$, by contrast, have no
tendency to form friendships with those ``nearby" them any more than with
anyone else in the world; it is as if their entire experience took place
through the Internet, with equal access to all others, and the ability to
search for others based on their attributes, but with all geographic
information hidden.

For simplicity, we model $A$ as a constant value throughout a society, rather
than giving it a different value for each agent (as it certainly has in
reality). Presumably the $A$ of the western world in 2017 is lower than the
$A$ of thirty years ago. One question this paper tries to answer is how this
might bear on ideological polarization.

\subsection{The reality of homophily}

One might expect that as the accessibility $A$ of a society increases, so
would the diversity of viewpoints that its citizens are exposed to. After all,
a higher $A$ means that more of a person's friendships will be drawn from the
entire global field, giving them a much broader range of exposure to people
with many different views and interests.

Arguing against that outcome, however, is an indisputable fact about human
nature: \textit{homophily}. One of the most reliable and long-standing
observations in social psychology, homophily simply refers to the tendency for
people to prefer others who are similar to them.\cite{mcpherson_birds_2001}
This is true across many different aspects of ``similarity," whether it be
race, age, religion, occupation, political affiliation, values, or common
interests. Given the choice of forming ties with several individuals, people
tend to choose the one(s) whom they perceive as being most like them. 

A greater number of choices of friends, therefore, may well lead to
\textit{less} diversity within one's social circle. As with accessibility, we
model the homophily $H$ of an entire society with a single value, $0 \leq H
\leq 1$, that influences friendship choices. In this case, an $H$ of 0.5 is
\textit{neutral}: agents in the model have no preference for or against
forming friendships with similar agents. At the extreme of $H=1$, agents will
always choose an agent similar to them, if possible, and at $H=0$, they will
always choose a \textit{dis}similar agent. (This situation could be termed
``heterophily.")

The interplay of these two parameters $A$ and $H$ and their impact on the
ideological polarization of a society is the subject of this paper.

\subsection{Defining a healthy society}

The term ``polarization" -- often with the modifier ``political" -- abounds in
recent discussion of the U.S. and other political
env\-ironments\cite{campbell_source_2016,french_were_2017,dimock_political_2014,mccarty_polarized_2016}.
It is nearly always used with a negative connotation. Claims that the degree
of polarization is increasing in western cultures have been substantiated in
some ways by academics (\textit{e.g.},
\cite{baldassarri_partisans_2008,prior_media_2013,abramowitz_new_2015})
although with caveats
(\cite{baldassarri_dynamics_2007,fiorina_political_2008,abrams_party_2015}).
Numerous studies have investigated how it takes root in social networking and
other online environments
\cite{mousavi_role_2014,conover_political_2011,adamic_political_2005,hargittai_cross-ideological_2008}.

Defining polarization, however, is somewhat tricky; one recent paper, in fact,
spelled out nine different possible
definitions\cite{bramson_disambiguation_2016}. Most often, the term is
associated with large \textit{differences} in views between subpopulations,
especially when those views are perceived as \textit{extreme}. If, when
responding to a question on a 6-point Likert scale, half the population
answered ``1" and the other half answered ``6," this would typically be viewed
as a ``polarized" population. If half answered ``3" and the other half ``4,"
on the other hand, or if equal numbers gave each of the six responses, that
would be seen as less polarized.

In this work, however, we look at a different form of ``polarization":
\textit{the extent to which adherents of one viewpoint tend to form social
connections only with others of that same viewpoint.} Whether the viewpoints
are themselves ``extreme" on any objective scale is irrelevant, as is the
percentage of individuals subscribing to each of the various viewpoints. What
matters for our purposes is whether the adherents of various views form
isolated pockets of communication, or ``echo chambers," rather than having
broad social connections with people of a variety of different opinions.

Under this interpretation, members of a society holding strong or even
``extreme" views is not a negative outcome. What is important is that the
members maintain fruitful dialogue with one another, and are continually
exposed to views different from their own. Members sequestering themselves
into ideological cliques is unhealthy. But members thoughtfully choosing
to retain their opinion even in the constant and active presence of others
articulating counterpoints to it is not.

We therefore use the social network's \textit{assortativity}
coefficient\cite{newman_mixing_2003} as the key measure of polarization. If we
model a social network as an undirected graph whose nodes each possess a
nominal ``ideology" attribute taken from a small set of possible ideologies,
the assortativity gives a measure of what fraction of the edges are between
likeminded nodes, compared to what we would expect if the edges were simply
dispersed at random.\footnote{Formally, the assortativity coefficient of a
graph is a value between -1 and 1 which is computed as follows. Let $e_{ij}$
be the fraction of all edges in the graph which connect an agent with ideology
$i$ and an agent with ideology $j$, where $i$ and $j$ range over all pairs of
possible ideologies. Let \textbf{e} be the matrix whose elements are $e_{ij}$,
$\textbf{x}^2$ indicate matrix multiplication, $\Tr \textbf{x}$ be the sum of
the main diagonal elements of \textbf{x}, and $\norm{\textbf{x}}$ be the sum
of the elements of the matrix \textbf{x}. The assortativity is then $\frac{\Tr
\textbf{e} - \norm{\textbf{e}^2}}{1 - \norm{\textbf{e}^2}}$. It has the value
1 when there is perfect assortative mixing (\textit{i.e.}, \textit{all} edges
are between nodes with the same ideology), 0 when there is no assortative
mixing (the ideology of the nodes has no bearing on whether they will be
connected), and a negative value when nodes tend to connect to others of a
\textit{different} ideology.}

