\documentclass[sigconf]{acmart}

\usepackage{booktabs} % For formal tables
\usepackage{mathtools} % For norm


% Copyright
%\setcopyright{none}
%\setcopyright{acmcopyright}
%\setcopyright{acmlicensed}
\setcopyright{rightsretained}
%\setcopyright{usgov}
%\setcopyright{usgovmixed}
%\setcopyright{cagov}
%\setcopyright{cagovmixed}

\DeclareMathOperator{\Tr}{Tr}
\DeclarePairedDelimiter{\norm}{\lVert}{\rVert}

% DOI
\acmDOI{10.475/123_4}

% ISBN
\acmISBN{123-4567-24-567/08/06}

%Conference
\acmConference[CSSSA'17]{The Computational Social Science Society of the
Americas}{October 2017}{Santa Fe, New Mexico USA} 
\acmYear{2017}
\copyrightyear{2017}

\acmPrice{15.00}


\begin{document}
\title[The twin impact of homophily and accessibility on ideological
polarization]{The twin impact of homophily and accessibility\\on ideological
polarization}
%\subtitle{Extended Abstract}


\author{Stephen Davies}
\affiliation{
  \institution{University of Mary Washington}
  \streetaddress{1301 College Avenue}
  \city{Fredericksburg} 
  \state{VA} 
  \postcode{22401}
}
\email{stephen@umw.edu}

% The default list of authors is too long for headers}
\renewcommand{\shortauthors}{S. Davies}


\begin{abstract}
We present an agent-based model to explore the causes of one aspect of
ideological polarization: the extent to which members of a society have social
ties only with those they agree with. Specifically, we look at two variables
that affect how an artificial social network structure is built:
\textit{homophily}, or the preference of individuals to form connections with
others of the same ``kind''; and \textit{accessibility}, or the ease with which
agents can form connections to others distant from it, as opposed to only
local agents in its immediate vicinity. Our model builds a graph according to
these two parameters, and then executes the classic Binary Voter Model (BVM)
process on it whereby connected nodes influence one another's opinions. We
find that counter to our original hypothesis, increasing the society's
accessibility decreases its polarization, especially for high levels of
homophily. Also, we discover that the rate at which agents form and dissolve
friendships during the simulation plays a nuanced role in the way the society
evolves.
\end{abstract}

%
% The code below should be generated by the tool at
% http://dl.acm.org/ccs.cfm
% Please copy and paste the code instead of the example below. 
%
\begin{CCSXML}
<ccs2012>
<concept>
<concept_id>10010147.10010341.10010349.10010355</concept_id>
<concept_desc>Computing methodologies~Agent / discrete models</concept_desc>
<concept_significance>500</concept_significance>
</concept>
</ccs2012>
\end{CCSXML}


\ccsdesc[500]{Computing methodologies~Agent / discrete models}

\keywords{Opinion Dynamics models, political polarization}


\maketitle

\section{Introduction}

\subsection{Two sources of friendship}

Consider for a moment the friendships you have had through which meaningful
mutual influence has taken place. In broad terms, these relationships can be
thought of as coming from two different kinds of sources. In the first kind,
you did not originally encounter the acquaintances by specifically seeking
them out -- they rather made an appearance in your life due to circumstances.
When you were a child, there were other children down the street, and fellow
students in your 1st-grade classroom. When you were older, there were
dormitory hall-mates, new neighbors in new neighborhoods, and co-workers. Some
of these people who by happenstance wandered into your field of view, you
formed meaningful friendships with. Others you did not. But the key point that
distinguishes them from the second group is that you did not discover them by
seeking them out based any of their attributes. Instead, a minuscule set of
people (out of all the people in the world) simply fell into your lap. And
from that set, you became meaningful friends with some.

The second group consists of those you encountered \textit{because} you had
something in common, and you discovered them because you deliberately sought
that something. Even before the advent of cheap, electronic communication this
occurred: people attended churches and synagogues, knowing they would find
others with similar worldviews. They joined rotary clubs, parent-teacher
organizations, and political parties, seeking out those with similar
interests. In the Internet age, this is multiplied tenfold. Anyone with any
view or interest, no matter how esoteric, can find a chat\-room, website, Google
group, subreddit, Twitter community, or other form of online clique devoted to
it. For the younger generation especially, this is an important group: recent
surveys suggest that a majority of American teens form meaningful friendships
online, and communicate with them daily, often never meeting the friend
physically.\cite{lenhart_teens_2015}

The key point is that some of your friends were ``chosen'' from a very tiny
subset of the people in the world who you couldn't help but run into. You
might not have had anything in particular in common with them, other than
geographical proximity. The others, however, you drew from a very large pool
-- essentially the entire online world.\footnote{Note that once a relationship
has been established, the partners may never again actively consider the
source of the friendship: a link has simply been made, regardless of the
cause, and thereafter results in mutual influence.} And the tools of the
Internet give you breathtaking precision with which to find and select such
friends.

\subsection{``Accessibility''}

Admittedly, these two sources of friendship are idealized points along a
continuum. Any individual relationship may have been formed due to some blend
of the two. Nevertheless, it is a useful abstraction, and suggests the
existence of a key parameter in modeling social networks: the relative
strength of the two sources in leading to friendships. Put another way: on
average, what fraction of a person's friendships develop as a result of
geographical proximity and happenstance encounters, versus being due to more
distant relationships where parties sought each other out based on some shared
attribute? 

In the language of this paper, we use the term \textbf{local} to describe
friendships of the first sort, and \textbf{global} for the second. In our
model we are concerned with the relative likelihood of friendships developing
from each of these two sources, and we denote as the society's
\textbf{accessibility} $A$ ($0 \leq A \leq 1$) the fraction of relationships
that have \textit{global} origination. A society with an $A$ of 0 presents no
way for its citizens to discover anyone outside the local social circle they
happened to inherit. Persons in a society with $A=1$, by contrast, have no
tendency to form friendships with those ``nearby'' them any more than with
anyone else in the world; it is as if their entire experience took place
through the Internet, with equal access to all others, and the ability to
search for others based on their attributes, but with all geographic
information hidden.

For simplicity, we model $A$ as a constant value throughout a society, rather
than giving it a different value for each agent (as it certainly has in
reality). Presumably the $A$ of the western world in 2017 is higher than the
$A$ of thirty years ago. One question this paper tries to answer is how this
might bear on ideological polarization.

\subsection{The reality of homophily}

One might expect that as the accessibility $A$ of a society increases, so
would the diversity of viewpoints that its citizens are exposed to. After all,
a higher $A$ means that more of a person's friendships will be drawn from the
entire global field, giving them a much broader range of exposure to people
with many different views and interests.

Arguing against that outcome, however, is an indisputable fact about human
nature: \textit{homophily}. One of the most reliable and long-standing
observations in social psychology, homophily simply refers to the tendency for
people to prefer others who are similar to them.\cite{mcpherson_birds_2001}
This is true across many different aspects of ``similarity,'' whether it be
race, age, religion, occupation, political affiliation, values, or common
interests. Given the choice of forming ties with several individuals, people
tend to choose the one(s) whom they perceive as being most like them. 

A greater number of choices of friends, therefore, may well lead to
\textit{less} diversity within one's social circle. As with accessibility, we
model the homophily $H$ of an entire society with a single value, $0 \leq H
\leq 1$, that influences friendship choices. In this case, an $H$ of 0.5 is
\textit{neutral}: agents in the model have no preference for or against
forming friendships with similar agents. At the extreme of $H=1$, agents will
always choose an agent similar to them, if possible, and at $H=0$, they will
always choose a \textit{dis}similar agent. (This situation could be termed
``heterophily.'')

The interplay of these two parameters $A$ and $H$ and their impact on the
ideological polarization of a society is the subject of this paper.

\subsection{Defining a healthy society}

The term ``polarization'' -- often with the modifier ``political'' -- abounds in
recent discussion of the U.S. and other political
env\-ironments\cite{campbell_source_2016,french_were_2017,dimock_political_2014,mccarty_polarized_2016}.
It is nearly always used with a negative connotation. Claims that the degree
of polarization is increasing in western cultures have been substantiated in
some ways by academics (\textit{e.g.},
\cite{baldassarri_partisans_2008,prior_media_2013,abramowitz_new_2015})
although with caveats
(\cite{baldassarri_dynamics_2007,fiorina_political_2008,abrams_party_2015}).
Numerous studies have investigated how it takes root in social networking and
other online environments
\cite{mousavi_role_2014,conover_political_2011,adamic_political_2005,hargittai_cross-ideological_2008}.

Defining polarization, however, is somewhat tricky; one recent paper, in fact,
spelled out nine different possible
definitions\cite{bramson_disambiguation_2016}. Most often, the term is
associated with large \textit{differences} in views between subpopulations,
especially when those views are perceived as \textit{extreme}. If, when
responding to a question on a 6-point Likert scale, half the population
answered ``1'' and the other half answered ``6,'' this would typically be viewed
as a ``polarized'' population. If half answered ``3'' and the other half
``4,''
on the other hand, or if equal numbers of people gave each of the six
responses, that would be seen as less polarized.

In this work, however, we look at a different form of ``polarization'':
\textit{the extent to which adherents of one viewpoint tend to form social
connections only with others of that same viewpoint.} Whether the viewpoints
are themselves ``extreme'' on any objective scale is irrelevant, as is the
percentage of individuals subscribing to each of the various viewpoints. What
matters for our purposes is whether the adherents of various views form
isolated pockets of communication, or ``echo chambers,'' rather than having
broad social connections with people of a variety of different opinions.

Under this interpretation, members of a society holding strong or even
``extreme'' views is not a negative outcome. What is important is that the
members maintain fruitful dialogue with one another, and are continually
exposed to views different from their own. Members sequestering themselves
into ideological cliques is unhealthy. But members thoughtfully choosing
to retain their opinion even in the constant and active presence of others
articulating counterpoints to it is not.

We therefore use the social network's \textit{assortativity}
coefficient\cite{newman_mixing_2003} as the key measure of polarization. If we
model a social network as an undirected graph whose nodes each possess a
nominal ``ideology'' attribute taken from a small set of possible ideologies,
the assortativity gives a measure of what fraction of the edges are between
likeminded nodes, compared to what we would expect if the edges were simply
dispersed at random.\footnote{Formally, the assortativity coefficient of a
graph is a value between -1 and 1 which is computed as follows. Let $e_{ij}$
be the fraction of all edges in the graph which connect an agent with ideology
$i$ and an agent with ideology $j$, where $i$ and $j$ range over all pairs of
possible ideologies. Let \textbf{e} be the matrix whose elements are $e_{ij}$,
$\textbf{x}^2$ indicate matrix multiplication, $\Tr \textbf{x}$ be the sum of
the main diagonal elements of \textbf{x}, and $\norm{\textbf{x}}$ be the sum
of the elements of the matrix \textbf{x}. The assortativity is then $\frac{\Tr
\textbf{e} - \norm{\textbf{e}^2}}{1 - \norm{\textbf{e}^2}}$. It has the value
1 when there is perfect assortative mixing (\textit{i.e.}, \textit{all} edges
are between nodes with the same ideology), 0 when there is no assortative
mixing (the ideology of the nodes has no bearing on whether they will be
connected), and a negative value when nodes tend to connect to others of a
\textit{different} ideology.}



\section{Related Work}

Opinion Dynamics (OD) models have a robust tradition, often traced to the
Binary Voter Model (BVM) of Holley and Liggett\cite{holley_ergodic_1975} and
Clifford and Sudbury\cite{clifford_model_1973}. OD models seek to reproduce
the phenomenon of individual agents forming opinions over time via mutual
influence, and to draw conclusions about the overall pattern of opinions that
may emerge in a society as a consequence of certain micro-behaviors.

Axelrod's work in this area\cite{axelrod_dissemination_1997} represented
agents with multiple discrete-valued attributes occupying a cellular grid.
Agents were more likely to influence one another when they had more attribute
values already in common, imitating what Axelrod called ``the fundamental
principle of human communication": that influence occurs more frequently
between people who perceive themselves as being already fairly alike. The
effect of influence in the model was to copy one of the differing attributes
from one agent to the other, further increasing their similarity. Among other
results, Axelrod demonstrated that as the \textit{range} of influence
increases (\textit{i.e.}, as agents are able to interact directly with other
agents 2, 3, 4, $\dots$ squares away), the degree of overall homogeneity in
the society increases. He measured homogeneity as the number of distinct,
geographically isolated clusters of agents with the same attribute values.

Axelrod's result might lead us to predict that polarization would
\textit{decrease} with accessibility, rather than increase. Shibani \textit{et
al.}\cite{shibanai_effects_2001} and Grieg\cite{greig_end_2002} found that
subtle changes to the model, however, can produce the opposite result. Too,
Flache and Macy\cite{flache_why_2006} concluded that Axelrod's original result
depended crucially on the opinions being discrete valued; when continuous
opinions were used, and adjustments could be made to them gradually,
polarization actually increased with range of influence.

One popular approach to modeling \textit{continuous} opinion dynamics is the
Bounded Confidence (BC) model (originally in
\cite{deffuant_mixing_2000,hegselmann_opinion_2002}). This assumes that agents
will be influenced only by the opinions of others that their own opinion is
already sufficiently ``close" to (\textit{i.e.}, within some threshold
$\epsilon$); other opinions are viewed as too extreme from one's own, and
therefore untrustworthy. (This is similar in spirit to Axelrod's ``fundamental
principle," but in the context of a single attribute, not an array of them.)
The result of such influence, when it does occur, is an averaging operation
that pulls each agent's opinion closer to the other. The BC mechanism is one
way of preventing a graph of continuous opinions from converging to absolute
homogeneity, as will happen if the averaging operation happens
unconditionally.

All of these studies inspired by Axelrod place agents on a rectangular grid.
Work has been done, too, on agents connected via a more general network/graph
structure, whether a complete graph, possibly with edge
weights\cite{deffuant_how_2002,degroot_reaching_1974}, or a general
graph\cite{dandekar_biased_2013}. Several of these studies have explored the
interplay between homophily and (various definitions of) polarization in the
context of continuous-valued opinions. Dandekar \textit{et
al.}\cite{dandekar_biased_2013} in particular use a variant of the
assortativity coefficient called the network degree index (for continuous
attributes). However, their work was still based on continuous attributes, and
they invoke a more complex opinion formation process than we do, incorporating
confirmation bias.

Recently, Gargiulo and Gandica\cite{gargiulo_role_2017} explored the
connection between homophily and polarization in the context of
continuous-valued opinions under a BC dynamic. To build the initial graph for
their simulation, they extend the well-known preferential attachment mechanism
\cite{barabasi_emergence_1999} to incorporate homophily: when a new node
chooses which existing node to connect to in the graph, it incorporates
information not only about the degree of existing nodes, but also about
the similarity of their opinions to its own. In this way, not only are
nodes with more neighbors more likely to be chosen for attachment (as in
\cite{barabasi_emergence_1999}, \cite{deffuant_mixing_2000}), but nodes with
similar opinions to the new node are also more likely. 

Gargiulo and Gandica discovered that under these assumptions, as homophily is
increased (\textit{i.e.}, as similarity is weighed more heavily than degree
when attaching new nodes), polarization -- measured as the number of distinct
opinion clusters at equilibrium -- \textit{decreases}. This counterintuitive
result can be explained as follows: when homophily is low, a node does not
form as many initial connections to nodes with similar opinions (within the
threshold $\epsilon$ of its own). Therefore, it is more likely that that node
will get ``stuck" during the bounded confidence process: since it can't find
many neighbors whose opinions seem plausible, it stubbornly sticks to its own.
Increasing homophily equips each node with more neighbors whose opinions are
close to its own, such that it can gradually be nudged towards the emerging
consensus.

Our work differs from these other studies in that we use a graph instead of a
cellular grid; discrete opinions rather than continuous; we implement a BVM
process instead of BC; and most importantly, we measure polarization as the
graph's \textit{assortativity} rather than as the number of distinct opinion
clusters or the extremity of views.


\section{The Model}

\section{THE MODEL}

As explained above, the BVM is deceptively simple. Each node in the graph is
initially assigned an opinion (say, 0 or 1) and updates it periodically by
copying the opinion of one of its graph neighbors. It has long been known that
such a system will reach consensus (uniformity of one opinion or the other)
under a wide variety of conditions (see, \textit{e.g.},
\cite{sood_voter_2005}). The probability that 0 (as opposed to 1) becomes the
dominant opinion as a function of the initial opinion distribution is known,
as is the expected number of iterations required to reach consensus for
various degree distributions of the graph.

Implementing this model as an agent-based simulation is straightforward. Yet
in reproducing these classical results en route to other work, we discovered
at least two subtle implementation choices that at first glance would appear
unimportant, and yet which impact the convergence time in striking fashion. We
present these not so much as important in their own right, but as exemplars of
a more general problem: implementation choices that a modeler takes for
granted may turn out to be critical to the behavior claimed for that model.

\subsection{Simulation variant 1}

\subsection{Simulation variant 2}





\section{Hypotheses}

\label{sec:hypos}

We form the following hypotheses about the above model's behavior.

\begin{description}

\item{\textbf{Hypothesis 1a ($H_{1a}$):}} For a given level of accessibility
$A$, the polarization $P$ \textit{of the initial graph} will increase with
homophily $H$.

\item{\textbf{Hypothesis 1b ($H_{1b}$):}} For a given $A$, $P$ will
\textit{continue} to increase with $H$ as the BVM process takes place on the
graph.

\item{\textbf{Hypothesis 2a ($H_{2a}$):}} For a given $H>.5$, the $P$ \textit{of
the initial graph} will increase with $A$.

\item{\textbf{Hypothesis 2b ($H_{2b}$):}} For a given $H$, $P$ will
\textit{continue} to increase with $A$ as the BVM process takes place on the
graph.

\item{\textbf{Hypothesis 3 ($H_3$):}} If the DynamicRebalancing policy is
enabled, and $H>.5$, $P$ will increase more rapidly during the BVM process,
and more rapidly with increasing $R$. This will be unaffected by $A$.

\end{description}

We expect $H_{1a}$ to hold simply because as $H$ increases, the preference for
nodes to attach to others of the same ideology increases, which should
increase assortativity. Less obviously, $H_{2a}$ should hold because when $A$
increases nodes have greater freedom of choice in who they connect to. Low
values of $A$ mean that as the graph is built, nodes will often be ``forced"
to choose a friend from the limited number present in their LocalAssociates
set, and will thus more often have no choice but to make a friend of a
different ideology. Thus we expect more cross-ideology friendships when $A$ is
low. (Obviously this only holds when the homophily is greater than .5,
indicating a preference for same-ideology nodes.)

The rationale behind $H_{1b}$ and $H_{2b}$ is as follows. When the initial
graph is constructed so as to be more polarized, the BVM will have more raw
material to work with in order to make it further polarized. We expect
increasing returns as the structure of the network, already built to put
like-minded nodes largely together, causes the remaining vestiges of local
disagreement to be snuffed out.

Finally, we predict $H_3$ because dynamically rebalancing should have a
strictly positive impact on assortativity (when $H>.5$). New edges will be
added to the graph, and old edges dropped, with preference to
forming/maintaining friendships with likeminded nodes. This should accelerate
the degree to which the graph becomes polarized, and the rate of acceleration
should increase when we add/remove friendships more often. We have no
\textit{a priori} reason to suspect that the $A$ used to form the initial
graph will come into play.


\section{Results}

\subsection{$H_{1a}$ and $H_{2a}$}

To verify $H_{1a}$ and $H_{2a}$, we need only consider the starting graph. We
generate a large number of initial graphs according to the process described
in section~\ref{sec:initialization}, using a range of values of $H$ and $A$,
and measure their assortativity. Figure~\ref{fig:initialPolarization} shows a
box plot of the results of generating 1000 such graphs for each combination of
six accessibility values and six homophily values. (For these and all other
results in this paper, $N$ was set to 50 agents and $I$ to 2 ideologies.) As
expected, the polarization of these initial graphs clearly increases with both
$H$ and $A$, establishing $H_{1a}$ and $H_{2a}$.

\begin{figure}
\centering
\includegraphics[width=1.0\columnwidth]{initialPolarization.png}
% Reproduce with:
% main_wide.py accessibility=0-1-.2 homophily=.5-.99-.1 suite=1000 
% seed=2959 num_iter=200 N=50 log_level=INFO
\caption{Polarizations of initial graphs, defined as nominal assortativity on
the ideology attribute, for varying levels of homophily $H$ and accessibility
$A$. The box for each pair of values represents 1000 randomly generated
starting graphs.}
\label{fig:initialPolarization}
\end{figure}

\subsection{$H_{1b}$ and $H_{2b}$}

Hypotheses $H_{1b}$ and $H_{2b}$ are a bit trickier to evaluate, since we seek
to discover whether the polarization of ``the graph" increases as a result
of the BVM. But of course the BVM produces an entire time series of graphs.
Several approaches are possible: we could take a snapshot of the graph at some
fixed number of iterations, and measure the assortativity at that point; we
could take the maximum (or minimum) assortativity anywhere in the sequence; we
could compute the mean assortativity of all graphs produced during the
process; \textit{etc.} We choose the last of these approaches here, but do not
begin computing the mean until the $50^{\text{th}}$ of 200 iterations. This
admittedly arbitrary boundary point is an attempt to measure the assortativity
only after the process has had a chance to emerge from a cold start. To
summarize, then: we choose as our measure of ``the polarization that the BVM
process induces" the mean assortativity of the graph at iterations 50 through
200 of the BVM process.

The result, quite surprising to us, is in the top half of
Figure~\ref{fig:meanNormPolarization}. For high levels of homophily
($H\geq.8$), additional accessibility does result in higher polarization. But
this effect seems to be less than it was on the initial graph, and in fact for
moderate levels of homophily ($.5 \leq H \leq .7$), higher accessibility
actually \textit{lowers} the polarization. 

To get a clearer view of the BVM's effect, we ``normalize" these
iterations-50-through-200 polarization values by subtracting each
\textit{initial} graph's polarization from them. In this manner, we isolate
the effect of the BVM process (section~\ref{sec:BVM}) from the effect of the
initial graph-construction process (section~\ref{sec:initialization}). The
result is the bottom plot in Figure~\ref{fig:meanNormPolarization}. Clearly,
for most values of homophily, the accessibility has a \textit{moderating}
effect, rather than an amplifying effect, on the graph's polarization. 

We thus not only fail to verify $H_{1b}$ and $H_{2b}$, but we radically refute
them. The opposite is true.

\begin{figure}
\centering
\includegraphics[width=1.0\columnwidth]{meanPolarization.png}
\includegraphics[width=1.0\columnwidth]{normPolarization.png}
% Reproduce with: (see above)

\caption{Top: mean polarization of the graph at iterations 50 through 200
of the BVM process. Bottom: the ``normalized" polarization, defined as the
difference between the mean polarization (top) and the initial polarization
(Figure~\ref{fig:initialPolarization}).}
\label{fig:meanNormPolarization}
\end{figure}

\subsection{$H_{3}$}

To evaluate $H_3$, we compare the normalized polarizations of simulations that
have \textbf{DynamicRebalancing} enabled with those that don't. The results
are presented in Figure~\ref{fig:rebalancing} (for clarity, we omit outliers
from the plots and show DynamicRebalancing at only a single rate, $R$=1.)
Interestingly, hypothesis $H_3$ is confirmed only for \textit{some} homphily
values. When $H\leq.6$, dynamic rebalancing acts as a moderating effect on
polarization. Only for values of .7 and above does it increase the
polarization relative to the initial graph, as predicted. 

This effect can possibly be explained by the small size of the graph (only 50
nodes). There are a fixed number of other nodes in the graph with the same
ideology as a given node $X$. Therefore, the more likeminded friends a node
has, the fewer additional nodes with that Ideology remain in the pool. If
forced to dissolve friendships often and replace them with new friends (which
happens for large values of $R$), there will be proportionately fewer
similarly likeminded nodes to replace that broken friendship with. Only if the
homophily is so high that $X$ aggressively cherry-picks likeminded neighbors
with strong preference will it be able to overcome this tendency prevent the
polarization from drifting lower. (Admittedly, this is only a conjecture about
the reasons for this puzzling effect.)

\begin{figure*}
\centering
\includegraphics[width=0.9\textwidth]{rebalancing.png}
\caption{Effect of the DynamicRebalancing policy on (normalized) mean
polarization. The blue boxplots are identical to those in the bottom half of
Figure~\ref{fig:meanNormPolarization}. The green boxplots use the same 
starting graphs, but with dynamic rebalancing enabled at a rate of $R$=1.}
\label{fig:rebalancing}
\end{figure*}
% Play with num ideologies?


\section{Conclusion}

The conclusion.




\appendix

\bibliographystyle{ACM-Reference-Format}
\bibliography{davies2017}

\end{document}
