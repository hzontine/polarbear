\documentclass[sigconf]{acmart}

\usepackage{booktabs} % For formal tables
\usepackage{mathtools} % For norm


% Copyright
%\setcopyright{none}
%\setcopyright{acmcopyright}
%\setcopyright{acmlicensed}
\setcopyright{rightsretained}
%\setcopyright{usgov}
%\setcopyright{usgovmixed}
%\setcopyright{cagov}
%\setcopyright{cagovmixed}

\DeclareMathOperator{\Tr}{Tr}
\DeclarePairedDelimiter{\norm}{\lVert}{\rVert}

% DOI
\acmDOI{10.475/123_4}

% ISBN
\acmISBN{123-4567-24-567/08/06}

%Conference
\acmConference[CSSSA'17]{The Computational Social Science Society of the
Americas}{October 2017}{Santa Fe, New Mexico USA} 
\acmYear{2017}
\copyrightyear{2017}

\acmPrice{15.00}


\begin{document}
\title{The twin impact of homophily and accessibility\\on ideological
polarization}
%\subtitle{Extended Abstract}


\author{Stephen Davies}
\affiliation{
  \institution{University of Mary Washington}
  \streetaddress{1301 College Avenue}
  \city{Fredericksburg} 
  \state{VA} 
  \postcode{22401}
}
\email{stephen@umw.edu}

% The default list of authors is too long for headers}
\renewcommand{\shortauthors}{S. Davies}


\begin{abstract}
We present an agent-based model to explore the causes of one aspect of
ideological polarization: the extent to which members of a society have social
ties only with those they agree with. Specifically, we look at two variables
that affect how an artificial social network structure is built:
\textit{homophily}, or the preference of individuals to form connections with
others of the same ``kind"; and \textit{accessibility}, or the ease with which
agents can form connections to others distant from it, as opposed to only
local agents in its immediate vicinity. Our model builds a graph according to
these two parameters, and then executes the classic Binary Voter Model (BVM)
process on it whereby connected nodes influence one another's opinions. We
find that counter to our original hypothesis, increasing the society's
accessibility decreases its polarization, especially for high levels of
homophily. Also, we discover that the rate at which agents form and dissolve
friendships during the simulation plays a nuanced role in the way the society
evolves.
\end{abstract}

%
% The code below should be generated by the tool at
% http://dl.acm.org/ccs.cfm
% Please copy and paste the code instead of the example below. 
%
\begin{CCSXML}
<ccs2012>
<concept>
<concept_id>10010147.10010341.10010349.10010355</concept_id>
<concept_desc>Computing methodologies~Agent / discrete models</concept_desc>
<concept_significance>500</concept_significance>
</concept>
</ccs2012>
\end{CCSXML}


\ccsdesc[500]{Computing methodologies~Agent / discrete models}

\keywords{Opinion Dynamics models, political polarization}


\maketitle

\section{Introduction}

\section{INTRODUCTION}

In this work, we consider Opinion Dynamics models, which seek to reproduce
the phenomenon of individual agents forming opinions over time via mutual
influence. The field boasts a large literature, full of claims about the
behavior of various such systems, some supported with mathematical proofs,
others sustained by empirical evidence from simulation results.

When implementing any agent-based simulation, the designer faces choices that
seem inconsequential --- ``In which order should I update the state variables
here?" ``Should agents be treated in a consistent order, or should they be
shuffled each time?" ``Do I use a while loop or a for loop in this function?"
``When simultaneously inserting several events into the event queue, which one
do I enqueue first?" \textit{Etc.} Yet despite their apparent unimportance, in
some cases, system behavior may actually hinge on what choice is made. If this
goes undiscovered, there is a risk that broad claims made about a general
class of system may in fact be contingent on certain non-obvious
implementation concerns.

Among other things, this underscores the importance of the ABM community
taking the time to reproduce results from the literature. If several
researchers, starting from the same conceptual description of a system,
independently build implementations that produce the same macro-level
behavior, this increases confidence that the claims made about the system are
indeed robust. If not, this may expose the presence of latent
implementation-dependent assumptions that should actually be promoted to the
model description proper, rather than being omitted and hence left to an
implementer. It is then worth considering whether the augmented model, with
the new assumptions made explicit, is still a reasonable abstraction of the
real-world system being studied.

In this paper, we look at two such simulation variants of possibly the
simplest of all Opinion Dynamics models: the original Binary Voter Model
\cite{holley_ergodic_1975,clifford_model_1973}. We argue that in neither case
is one implementation choice obviously preferable to the other, and yet what
may seem a matter of indifference actually has a profound effect on the
runtime characteristics of the model -- in this case, the convergence time to
consensus.



\section{Related Work}

Opinion Dynamics (OD) models have a robust tradition, often traced to the
Binary Voter Model (BVM) of Holley and Liggett\cite{holley_ergodic_1975} and
Clifford and Sudbury\cite{clifford_model_1973}. OD models seek to reproduce
the phenomenon of individual agents forming opinions over time via mutual
influence, and to draw conclusions about the overall pattern of opinions that
may emerge in a society as a consequence of certain micro-behaviors.

Axelrod's work in this area\cite{axelrod_dissemination_1997} represented
agents with multiple discrete-valued attributes occupying a cellular grid.
Agents were more likely to influence one another when they had more attribute
values already in common, imitating what Axelrod called ``the fundamental
principle of human communication": that influence occurs more frequently
between people who perceive themselves as being already fairly alike. The
effect of influence in the model was to copy one of the differing attributes
from one agent to the other, further increasing their similarity. Among other
results, Axelrod demonstrated that as the \textit{range} of influence
increases (\textit{i.e.}, as agents are able to interact directly with other
agents 2, 3, 4, $\dots$ squares away), the degree of overall homogeneity in
the society increases. He measured homogeneity as the number of distinct,
geographically isolated clusters of agents with the same attribute values.

Axelrod's result might lead us to predict that polarization would
\textit{decrease} with accessibility, rather than increase. Shibani \textit{et
al.}\cite{shibanai_effects_2001} and Grieg\cite{greig_end_2002} found that
subtle changes to the model, however, can produce the opposite result. Too,
Flache and Macy\cite{flache_why_2006} concluded that Axelrod's original result
depended crucially on the opinions being discrete valued; when continuous
opinions were used, and adjustments could be made to them gradually,
polarization actually increased with range of influence.

One popular approach to modeling \textit{continuous} opinion dynamics is the
Bounded Confidence (BC) model (originally in
\cite{deffuant_mixing_2000,hegselmann_opinion_2002}). This assumes that agents
will be influenced only by the opinions of others that their own opinion is
already sufficiently ``close" to (\textit{i.e.}, within some threshold
$\epsilon$); other opinions are viewed as too extreme from one's own, and
therefore untrustworthy. (This is similar in spirit to Axelrod's ``fundamental
principle," but in the context of a single attribute, not an array of them.)
The result of such influence, when it does occur, is an averaging operation
that pulls each agent's opinion closer to the other. The BC mechanism is one
way of preventing a graph of continuous opinions from converging to absolute
homogeneity, as will happen if the averaging operation happens
unconditionally.

All of these studies inspired by Axelrod place agents on a rectangular grid.
Work has been done, too, on agents connected via a more general network/graph
structure, whether a complete graph, possibly with edge
weights\cite{deffuant_how_2002,degroot_reaching_1974}, or a general
graph\cite{dandekar_biased_2013}. Several of these studies have explored the
interplay between homophily and (various definitions of) polarization in the
context of continuous-valued opinions. Dandekar \textit{et
al.}\cite{dandekar_biased_2013} in particular use a variant of the
assortativity coefficient called the network degree index (for continuous
attributes). However, their work was still based on continuous attributes, and
they invoke a more complex opinion formation process than we do, incorporating
confirmation bias.

Recently, Gargiulo and Gandica\cite{gargiulo_role_2017} explored the
connection between homophily and polarization in the context of
continuous-valued opinions under a BC dynamic. To build the initial graph for
their simulation, they extend the well-known preferential attachment mechanism
\cite{barabasi_emergence_1999} to incorporate homophily: when a new node
chooses which existing node to connect to in the graph, it incorporates
information not only about the degree of existing nodes, but also about
the similarity of their opinions to its own. In this way, not only are
nodes with more neighbors more likely to be chosen for attachment (as in
\cite{barabasi_emergence_1999}, \cite{deffuant_mixing_2000}), but nodes with
similar opinions to the new node are also more likely. 

Gargiulo and Gandica discovered that under these assumptions, as homophily is
increased (\textit{i.e.}, as similarity is weighed more heavily than degree
when attaching new nodes), polarization -- measured as the number of distinct
opinion clusters at equilibrium -- \textit{decreases}. This counterintuitive
result can be explained as follows: when homophily is low, a node does not
form as many initial connections to nodes with similar opinions (within the
threshold $\epsilon$ of its own). Therefore, it is more likely that that node
will get ``stuck" during the bounded confidence process: since it can't find
many neighbors whose opinions seem plausible, it stubbornly sticks to its own.
Increasing homophily equips each node with more neighbors whose opinions are
close to its own, such that it can gradually be nudged towards the emerging
consensus.

Our work differs from these other studies most significantly in that we
measure ``polarization" as the graph's assortativity, rather than as the
number of distinct opinion clusters that emerge, or the extremity of views.
Dandekar \textit{et al.}'s is the only work we are aware of that focuses on
something akin to the assortativity coefficient, and their model is quite
different: it uses continuous-valued attributes, weighted edges, confirmation
bias, a starting graph based on a stochastic block model, and a BC-like
process of repeated opinion averaging. Ours instead models a BVM process on
discrete attributes, with the starting graph produced through a local/global
morphogenesis process as previously described.



\section{The Model}

\label{sec:model}

We present the model using an abbreviated version of the ODD protocol
\cite{polhill_using_2008}.

\subsection{Purpose}

This abstract agent-based model simulates an evolving social network of agents
that each possess an ``ideology" attribute representing some opinion. As it
evolves, nodes are randomly chosen to influence their neighbors by propagating
their ideologies to them. The ``polarization" of the network, measured as the
tendency of nodes to be connected to others with the same ideology, will
change as this process takes place. 

The purpose of the model is to investigate the effects on polarization of two
parameters that affect how the initial graph is constructed. One,
``\textbf{accessibility}," controls the set of possible neighbors that a node
chooses from each time it adds a friendship. When accessibility is low,
neighbors will more often be chosen from a small (random) pool that is
initially available to a node. When it is high, neighbors will more often be
chosen from the entire network at large. Accessibility thus models the extent
to which a society's citizens form \textbf{local} friendships (\textit{e.g.},
geographical neighbors) versus \textbf{global} ones (\textit{e.g.}, over the
Internet).

The second parameter, ``\textbf{homophily}," controls the strength of each
node's preference to attach to other nodes with the same ideology. When it is
high, nodes will almost always form connections to nodes that have the same
ideology, if possible. This is true regardless of whether they are formed
locally or globally. When low, nodes prefer to attach to nodes with
\textit{different} ideologies, and when medium, they are indifferent.

With this model, we hope to gain general insight into how societies that
provide different means of communication and discovery, and societies that
encourage different levels of tolerance for opposing opinions, may differ in
the prevalence of ``echo chambers" whereby dissenting views are rarely heard.

\subsection{Entities, State Variables and Scales}

The entities in the model are Agents, and have the following attributes:

\begin{description}
\itemsep.1em
\item[ID] A unique ID number.

\item[Ideology] One of a discrete set of possible ideologies, represented as
integers $0, 1, \dots I$. The ideology of an Agent will change during the
simulation, possibly many times, as it is influenced by its neighbors.

\item[Friendships] A set of references to other Agent entities with whom this
Agent has a social connection. The entire set of Agents and their Friendships
form an \textit{undirected} graph: if Agent $X$ is friends with Agent $Y$,
Agent $Y$ is also friends with Agent $X$.  Once the initial graph is built,
each Agent's set of Friendships is fixed over the lifetime of the model
(unless the \textbf{DynamicRebalancing} Policy is enabled; see below.)

\item[LocalAssociates] A set of references to other Agent entities with whom
this Agent \textit{may} form a friendship when it chooses locally (see below).
Conceptually, these represent the (small, relative to the whole population)
group of other people to whom an agent is geographically proximate. Some of
these may actually become the agent's friends; perhaps many or all of them if
the society has low accessibility. Importantly, once the initial graph is
constructed, the LocalAssociates attribute is discarded and no longer used in
the simulation.

\end{description}



\subsection{Process Overview and Scheduling}
\label{sec:BVM}

Once the initial graph is built (see ``Initialization," below), we carry out
the standard Binary Voter Model (BVM) process on it, using the
\textbf{selection with replacement} and \textbf{neighbor influences node}
variants (see \cite{davies_computational_2016}.) At each of $T$ iterations:

\begin{enumerate}
\itemsep.1em
\item An agent $X$ is chosen at random from the entire graph.
\item One of its friends $Y$ is chosen at random from its set of Friendships.
(If the agent has no Friendships, skip this iteration.)
\item If $Y$'s current ideology is different than $X$'s, copy $Y$'s to $X$'s.
\item If the \textbf{DynamicRebalancing} Policy is enabled, and its rate $R$
is such that it should take place now, carry out the
\textbf{DynamicallyRebalance} on agent $X$ (see below).
 
\item Compute and store the graph's assortativity coefficient with respect to
Ideology (using the \texttt{assortativity\_nominal} function from the
\texttt{igraph} Python package\cite{csardi_igraph_2006}). \end{enumerate}


\subsection{Initialization (morphogenesis)}
\label{sec:initialization}

When the simulation begins, generate an undirected Erdos-Renyi random
graph\cite{erdos_evolution_1960} with $N$ nodes and probability of edge
connection $p$. Call this the \textbf{LocalAssociatesGraph}. Each node
represents an Agent, and the edges of this graph constitute its
\textbf{LocalAssociates}. With uniform probability, assign each node an
initial Ideology from the set of $I$ ideologies.

Then, generate the \textbf{Friendships} graph as follows:

\begin{enumerate}
\item Create an empty \textbf{Friendships} graph with $N$ nodes (agents).
\item For each node $X$ (in ID order), generate $a_X$ friendship connections
to other nodes, where $a_X$ is the degree of node $X$ in the
\textbf{LocalAssociatesGraph}. Choose the node for each connection as follows:
    \begin{itemize}
    \item With probability $A$ (the accessibility parameter), select a node
(without replacement) from $X$'s \textbf{LocalAssociates}, weighted by $H$
(the homophily parameter, see below).
    \item With probability $1-A$, select a node (without replacement) from
\textit{all} nodes, weighted by $H$.
    \end{itemize}
    In either case, if there are no nodes available with whom $X$ is not
already friends, skip the step.
\end{enumerate}

``Weighted by $H$" means that candidate nodes whose ideologies are the same as
node $X$'s are assigned an (unnormalized) probability of $H$ to be selected,
and candidate nodes of different ideologies are assigned $1-H$. All
probabilities are then normalized to sum to 1, and a node is chosen.\footnote{
Therefore, if there are $c_X$ candidate nodes for $X$ to connect to, $s_X$ of
whose ideologies match $X$, the probability that each same-ideology node will
be chosen is $\alpha H$ and the probability for each different-ideology node
is $\alpha (1-H)$, where $\alpha$ is the normalizing constant $\frac{1}{s_X H
+ (c_X-s_X)(1-H)}$.}



\subsection{Submodels}

\textit{\textbf{DynamicRebalancing}.} If this optional Policy is enabled, a
``rebalance rate" parameter $R$ ($0 \leq R \leq 1$) controls how frequently
the \textbf{DynamicallyRebalance} submodel is executed. $R$=1 means it will
execute \textit{every} iteration of the main simulation loop (immediately
after copying the neighbor's ideology.) $R$=0 means it will never execute, and
any value in between means it will execute some fraction of the iterations.

When it executes for a node $X$, the \textbf{DynamicallyRebalance} process is
as follows:
\begin{enumerate}
\item Perform \textit{one} of the following:
\itemsep.1em
\begin{itemize}
\item With probability .5, connect a new node to $X$, chosen (without
replacement) from all other nodes, weighted by the homophily $H$ (exactly as
above).\footnote{In the unlikely event that $X$ is already connected to all
other nodes, skip this step.}
\item With probability .5, \textit{dis}connect a node from $X$ (breaking the
friendship), chosen from its current Friends, weighted by $1-H$. This means
that for $H>.5$, nodes with the same ideology as $X$ are \textit{less} likely
to be chosen than nodes with different ideologies.
\end{itemize}
\end{enumerate}


All of the numerical parameters are configurable. The simulation's default
values for them are $T$=500, $I$=3, $N$=50, $p$=.06, $A$=.5, $H$=.7, $R$=0,
and \textbf{DynamicRebalancing} disabled. The code is written in Python, is
open source, and available at
\url{https://github.com/hzontine/polarbear/tree/master/wide}.


\section{Hypotheses}

\label{sec:hypos}

We form the following hypotheses about the above model's behavior.

\begin{description}

\item{\textbf{Hypothesis 1a ($H_{1a}$):}} For a given level of accessibility
$A$, the polarization $P$ \textit{of the initial graph} will increase with
homophily $H$.

\item{\textbf{Hypothesis 1b ($H_{1b}$):}} For a given $A$, $P$ will
\textit{continue} to increase with $H$ as the BVM process takes place on the
graph.

\item{\textbf{Hypothesis 2a ($H_{2a}$):}} For a given $H$>.5, the $P$ \textit{of
the initial graph} will increase with $A$.

\item{\textbf{Hypothesis 2b ($H_{2b}$):}} For a given $H$>.5, $P$ will
\textit{continue} to increase with $A$ as the BVM process takes place on the
graph.

\item{\textbf{Hypothesis 3 ($H_3$):}} If the \textbf{DynamicRebalancing}
policy is enabled, and $H$>.5, $P$ will increase more rapidly during the BVM
process, and more rapidly with increasing $R$. This will be unaffected by $A$.

\end{description}

We expect $H_{1a}$ to hold simply because as $H$ increases, the preference for
nodes to attach to others of the same ideology increases, which should
increase assortativity. Less obviously, $H_{2a}$ should hold because when $A$
increases nodes have greater freedom of choice in who they connect to. Low
values of $A$ mean that as the graph is built, nodes will often be ``forced"
to choose a friend from the limited number present in their LocalAssociates
set, and will thus more often have no choice but to make a friend of a
different ideology. Thus we expect more cross-ideology friendships when $A$ is
low. (Obviously this only holds when the homophily is greater than .5,
indicating a preference for same-ideology nodes.)

The rationale behind $H_{1b}$ and $H_{2b}$ is as follows. When the initial
graph is constructed so as to be more polarized, the BVM will have more raw
material to work with in order to make it further polarized. We expect
increasing returns as the structure of the network, already built to put
like-minded nodes largely together, causes the remaining vestiges of local
disagreement to be snuffed out.

Finally, we predict $H_3$ because dynamically rebalancing should have a
strictly positive impact on assortativity (when $H>.5$). New edges will be
added to the graph, and old edges dropped, with preference to
forming/maintaining friendships with likeminded nodes. This should accelerate
the degree to which the graph becomes polarized, and the rate of acceleration
should increase when we add/remove friendships more often. We have no
\textit{a priori} reason to suspect that the $A$ used to form the initial
graph will come into play.


\section{Results}

To verify $H_{1a}$ and $H_{2a}$, we generate a large number of initial graphs
according to the process described in section~\ref{sec:initialization}, using
a range of values of $H$ and $A$, and measure their assortativity.
Figure~\ref{fig:initialPolarization} shows a box plot of the results of
generating 1000 such graphs for each combination of six accessibility values
and six homophily values. (For these and all other results in this paper, $N$
was set to 50 agents and $I$ to 2 ideologies.) As expected, the polarization
of these initial graphs clearly increases with both $H$ and $A$, establishing
$H_{1a}$ and $H_{2a}$.

\begin{figure}
\centering
\includegraphics[width=1.0\columnwidth]{initialPolarization.png}
% Reproduce with:
% main_wide.py accessibility=0-1-.2 homophily=.5-.99-.1 suite=1000 
% seed=2959 num_iter=200 N=50 log_level=INFO
\caption{Polarizations of initial graphs, defined as nominal assortativity on
the ideology attribute, for varying levels of homophily $H$ and accessibility
$A$. The box for each pair of values represents 1000 randomly generated
starting graphs.}
\label{fig:initialPolarization}
\end{figure}


Hypotheses $H_{1b}$ and $H_{2b}$ are a bit trickier to evaluate, since we seek
to discover whether the polarization of ``the graph" increases as a result
of the BVM. But of course the BVM produces an entire time series of graphs.
Several approaches are possible: we could take a snapshot of the graph at some
fixed number of iterations, and measure the assortativity at that point; we
could take the maximum (or minimum) assortativity anywhere in the sequence; we
could compute the mean assortativity of all graphs produced during the
process; \textit{etc.} We choose the last of these approaches here, but do not
begin computing the mean until the $50^{\text{th}}$ iteration of 200. This
admittedly arbitrary boundary point allows us to measure the assortativity
only after the process has had a chance to emerge from a cold start. To
summarize, then: we choose as our measure of ``the polarization that the BVM
process induces" the mean assortativity of the graph at iterations 50 through
200 of the BVM process.

The result, quite surprising to us, is in the top half of
Figure~\ref{fig:meanNormPolarization}. For high levels of homophily
($H\geq.8$), additional accessibility does result in higher polarization. But
this effect seems to be less than it was on the initial graph, and in fact for
moderate levels of homophily ($.5 \leq H \leq .7$), higher accessibility
actually \textit{lowers} the polarization. 

To get a clearer view of the BVM's effect, we ``normalize" these
iterations-50-through-200 polarization values by subtracting each
\textit{initial} graph's polarization from them. In this manner, we isolate
the effect of the BVM process (section~\ref{sec:BVM}) from the effect of the
initial graph-construction process (section~\ref{sec:initialization}). The
result is the bottom plot in Figure~\ref{fig:meanNormPolarization}. Clearly,
for most values of homophily, the accessibility has a \textit{moderating}
effect, rather than an amplifying effect, on the graph's polarization. We thus
not only fail to verify $H_{1b}$ and $H_{2b}$, but we radically refute them:
the opposite is true.

\begin{figure}
\centering
\includegraphics[width=1.0\columnwidth]{meanPolarization.png}
\includegraphics[width=1.0\columnwidth]{normPolarization.png}
% Reproduce with: (see above)

\caption{(1) Top: mean polarization of the graph at iterations 50 through 200
of the BVM process. (2) Bottom: the ``normalized" polarization, defined as the
difference between the mean polarization (top) and the initial polarization
(Figure~\ref{fig:initialPolarization}).}
\label{fig:meanNormPolarization}
\end{figure}

% Play with num ideologies?


\section{Discussion and Conclusion}

\section{CONCLUSION}

With this brief paper we hope simply to raise an alarm bell in the community
of agent-based social simulation. In the midst of the (admirable) efforts to
create abstract models reflective of reality, we have discovered that it is
easy for modelers to overlook the impact that presumably irrelevant
implementation choices may have on a simulation's behavior. The two examples
we raise in this article were ones we encountered in our own work, and which
surprised us a great deal. The silver lining was that this led us to redouble
our efforts to think through all aspects of our implementation a second time,
with an eye to unmasking choices we may have made without realizing that they
were important choices.

Only two remedies occur to us. The first is for ABM researchers to develop
this kind of scrutinizing mindset as a habit. Moving from the modeling phase
to the implementation phase of a project may not always just be a matter of
``cranking it out." It would be wise to adopt a cautious attitude and to
routinely second-guess every line of code: is the implementation decision I am
about to make a natural consequence of what the model demands? Or is this an
unwitting subjective choice which may nudge the system into giving a certain
outcome and not another, despite the fact that the model itself does not
prescribe either?

The second corrective measure is simply for the community to regularly
reproduce the work of peers. We hear much about ``reproducible research" in
the ABM community these days, but perhaps less about actually
\textit{reproducing} such research. If models are theoretically reproducible
but in practice rarely are, the dangers of unrealized assumptions lurking in
computational social science results will have few opportunities to be
exposed. For our part, we have resolved to more consistently attempt to
replicate the results of publications in our areas of interest in the future,
in order to learn more about the details of these models, and to help our
community produce even stronger, more transparent science.

\section*{ACKNOWLEDGEMENTS}

We are grateful to Dr.~David Marchette and to the UMW Summer Science Institute
for ideas, energy, and financial support.




\appendix

\bibliographystyle{ACM-Reference-Format}
\bibliography{davies2017}

\end{document}
