
To verify $H_{1a}$ and $H_{2a}$, we generate a large number of initial graphs
according to the process described in section~\ref{sec:initialization}, using
a range of values of $H$ and $A$, and measure their assortativity.
Figure~\ref{fig:initialPolarization} shows a box plot of the results of
generating 1000 such graphs for each combination of six accessibility values
and six homophily values. (For these and all other results in this paper, $N$
was set to 50 agents and $I$ to 2 ideologies.) As expected, the polarization
of these initial graphs clearly increases with both $H$ and $A$, establishing
$H_{1a}$ and $H_{2a}$.

\begin{figure}
\centering
\includegraphics[width=1.0\columnwidth]{initialPolarization.png}
% Reproduce with:
% main_wide.py accessibility=0-1-.2 homophily=.5-.99-.1 suite=1000 
% seed=2959 num_iter=200 N=50 log_level=INFO
\caption{Polarizations of initial graphs, defined as nominal assortativity on
the ideology attribute, for varying levels of homophily $H$ and accessibility
$A$. The box for each pair of values represents 1000 randomly generated
starting graphs.}
\label{fig:initialPolarization}
\end{figure}


Hypotheses $H_{1b}$ and $H_{2b}$ are a bit trickier to evaluate, since we seek
to discover whether the polarization of ``the graph" increases as a result
of the BVM. But of course the BVM produces an entire time series of graphs.
Several approaches are possible: we could take a snapshot of the graph at some
fixed number of iterations, and measure the assortativity at that point; we
could take the maximum (or minimum) assortativity anywhere in the sequence; we
could compute the mean assortativity of all graphs produced during the
process; \textit{etc.} We choose the last of these approaches here, but do not
begin computing the mean until the $50^{\text{th}}$ iteration of 200. This
admittedly arbitrary boundary point allows us to measure the assortativity
only after the process has had a chance to emerge from a cold start. To
summarize, then: we choose as our measure of ``the polarization that the BVM
process induces" the mean assortativity of the graph at iterations 50 through
200 of the BVM process.

The result, quite surprising to us, is in the top half of
Figure~\ref{fig:meanNormPolarization}. For high levels of homophily
($H\geq.8$), additional accessibility does result in higher polarization. But
this effect seems to be less than it was on the initial graph, and in fact for
moderate levels of homophily ($.5 \leq H \leq .7$), higher accessibility
actually \textit{lowers} the polarization. 

To get a clearer view of the BVM's effect, we ``normalize" these
iterations-50-through-200 polarization values by subtracting each
\textit{initial} graph's polarization from them. In this manner, we isolate
the effect of the BVM process (section~\ref{sec:BVM}) from the effect of the
initial graph-construction process (section~\ref{sec:initialization}). The
result is the bottom plot in Figure~\ref{fig:meanNormPolarization}. Clearly,
for most values of homophily, the accessibility has a \textit{moderating}
effect, rather than an amplifying effect, on the graph's polarization. We thus
not only fail to verify $H_{1b}$ and $H_{2b}$, but we radically refute them:
the opposite is true.

\begin{figure}
\centering
\includegraphics[width=1.0\columnwidth]{meanPolarization.png}
\includegraphics[width=1.0\columnwidth]{normPolarization.png}
% Reproduce with: (see above)

\caption{(1) Top: mean polarization of the graph at iterations 50 through 200
of the BVM process. (2) Bottom: the ``normalized" polarization, defined as the
difference between the mean polarization (top) and the initial polarization
(Figure~\ref{fig:initialPolarization}).}
\label{fig:meanNormPolarization}
\end{figure}

% Play with num ideologies?
