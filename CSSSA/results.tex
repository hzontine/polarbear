
\section{RESULTS}

\subsection{Simulation Variant 1}
*Holley vs. Zontine*

Everyone gets a turn. vs. Random





\subsection{Simulation Variant 2}

*Victim vs. Influencer*

Such a trivial implementation choice about the direction of information flow 
in a pairwise interaction yields unexpected results.

We define victims as agents who either, once an iteration or randomly, always 
adopt the opinion of a neighbor if their opinions happen to disagree. On 
average, when the agents are victims, it takes fewer than half the number
of iterations to reach a complete consensus.

Our inital thought was to model our simulation the other way.

But after noticing how Holley and Clifford's implemenation.

Our iterpretation of the 

When each agent is set up as the influencer, he will only have impact over
agents' opinions with whom he is connected. Therefore, agents with a higher
degree of edges are influenced the most often. An agent is restricted to only 
influence a subset of the nodes (i.e. his neighbors) in the graph on each 
cycle. This minor implementation choice is effectively biased toward choosing
higher-degree nodes. As a result, nodes with lower-degrees have significantly
less probability of changing. Acting in a similar fashion as stubborn nodes 
(Yildiz), we observed a slower, if at all, rate of convergence time.


What about random victim vs. influencer for each interaction?


\subsection{The Box}
