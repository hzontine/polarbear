
\section{RESULTS}

\subsection{Simulation Variant 1}





\subsection{Simulation Variant 2}

*Victim vs. Influencer*

Such a trivial implementation choice about the direction of information flow 
in a pairwise interaction yields unexpected results. On average, when the 
agents are victims, complete consensus is reached faster then when the agents
are influencers. Our iterpretation of this phenomenon pertains to the degree
of neighbors an agent has. When each agent is always doing the influencing, 
he will only have impact over agents' opinions with whom he is connected. 
Therefore, agents with a higher degree of edges are influenced the most
often. An agent is restricted to only influence a subset of the nodes
(i.e. his neighbors) in the graph on each cycle. This minor implementation
choice is effectively biased toward choosing higher-degree nodes. As a result,
nodes with lower-degrees have significantly less probability of changing. 
Acting in a similar fashion as stubborn nodes (Yildiz), we observed a slower, 
if at all, rate of convergence time.

Perhaps a resolution could be found in random sampling whether the agent is chosen 
to be the victim or influencer for every iteration?


\subsection{The Box}


A: Victim and Random


B: Victim and Not Random
- fastest

C: Influencer and Random
- slowest


D: Influencer and Not Random
