
\section{OPINION DYNAMICS MODELS}

Published independently by \cite{clifford_model_1973} and
\cite{holley_ergodic_1975} in the 1970s, the Binary Voter Model (BVM) laid the
initial foundation from which many other Opinion Dynamics models have been
constructed. The BVM, which represents an individual's opinion as a single
binary value, was intentionally simplified in several ways for the sake of
deriving an analytical solution. One simplification, for instance, is that
agents are distributed on a regular lattice rather than an arbitrary graph.
Periodically, a randomly chosen agent adopts the opinion of a neighboring
agent if the two opinions differ. Importantly, over time the BVM will always
reach uniformity of opinion, according to \cite[ch.~14]{aldous-fill-2014}.

Numerous models for the Opinion Dynamics phenomenon have been proposed,
varying in a multitude of ways, for instance:

\begin{itemize}
\item the way opinions are represented:
    \begin{itemize}
    \item discrete, with two or more distinct options \cite{follmer_random_1974,yildiz_discrete_2011} or
    continuous, typically a value between 0 and 1 \cite{ghaderi_opinion_2012,weisbuch_interacting_2001}

    \item single \cite{weisbuch_dynamical_1999} or
    multiple, where opinions on multiple different topics are considered \cite{deffuant_mixing_2000,sirbu_opinion_2013}

    \item expressed (most models) or latent, in which the true value is possibly hidden from other agents \cite{friedkin_social_1990}
    \end{itemize}

\item how agents encounter each other:
    \begin{itemize}
    \item randomly from the whole population \cite{hegselmann_opinion_2002}, or % cite DW too
    neighbors in a social network \cite{clifford_model_1973,holley_ergodic_1975}
    \item pairwise (most models) or in groups \cite{degroot_reaching_1974}
    
    \end{itemize}
\item how influence takes place:
    \begin{itemize}
    \item copying another agent's opinion \cite{holley_ergodic_1975}
    \item averaging their opinion with one's own \cite{degroot_reaching_1974}
    \item ``disagreement" processes where opinions diverge rather than
converge\cite{sirbu_opinion_2013});
    \end{itemize}

\end{itemize}

Other innovations have been studied as well. 
\cite{yildiz_discrete_2011} expanded the BVM by adding a binary
``stubbornness" attribute to each agent. Stubborn agents never update their
opinion and 
% vvv ? reword?
others would often be changing depending upon whom they interacted with. They
discovered that the addition of only a few stubborn individuals always
results in a graph polarized by opinion (\textit{i.e.}, non-consensus).

\cite{weisbuch_interacting_2001} designed a model with continuous opinion
values, where agents only adopt the opinion of a neighbor when the difference
between their current opinions is below a fixed threshold value. (This is
termed ``bounded confidence.") \cite{ghaderi_opinion_2012} further expanded
this model by considering degrees of stubbornness; agents differ in the degree
to which they are biased in favor of their initial opinion. Both of these
approaches were motivated by a desire to prevent the model from always
reaching uniformity of opinion in equilibrium.

As indicated above, some researchers
have explored models where agents have multiple opinion values.
\cite{deffuant_mixing_2000} gave each agent a vector of discrete (binary,
actually) opinions on different subjects. Agents aggregate opinions through pairwise
evaluation, 
% vvv ? how can you "slightly shift" a binary value?
slightly shifting an agent's opinion after looping through all possible
pairings within the population. After more than 1000 interactions among the
agents, the researchers discovered orthogonalization of opinions, no
polarization, and no correlation between the opinion vectors.

% vvv I say we scrap this one unless we're going to say more
%\cite{friedkin_social_1990} modeled an analytical approach which makes
%simplifying assumptions about the process of opinion change. Latent vs.
%expressed?

Finally, note that while the BVM restricts agents to only interact with their
neighbors, other models have simulated interactions where agents consider the
opinions of a group of other agents in the graph.
\cite{hegselmann_opinion_2002} (HK) constructed a model where agents encounter
others randomly from the whole population. The HK model uses bounded
confidence on a continuous opinion scale. Each randomly selected agent
aggregates all other agent opinions within his confidence bound and considers
the group average for shifting his opinion. In the DeGroot Model
\cite{degroot_reaching_1974}, an agent updates his opinion to be a weighted
average of his own opinion and the opinions of his neighbors.
 
*** Sirbu ***





