
\section{RELATED WORK}

Opinion Dynamics is the study of how individuals formulate and change their personal 
opinions based on interactions with others. Published independently by Clifford and 
Holley in the 1970s, the Binary Voter Model (BVM) laid the inital foundation from 
which other models were constructed from. The BVM, which represents opinions as a 
single binary value, was intentionally simplified for the sake of an analytical 
solution. Periodically, a randomly chosen agent adopts the opinion of a directly 
connected agent if the two opinions differ. Over time, the BVM will always reach 
uniformity of opinion, according to Aldous and Fill. Yildiz et al. redesigned the
BVM by adding binary stubbornness attribute to each agent. Some would never update
their opinion and others would often be changing depending whom they interacted with.
They discovered that the addition of only a few stubborn individuals always resulted 
in a graph polarized by opinion.

Other classical models have been proposed where opinions are represented by continuous 
values, typically between 0 and 1. Weisbuch et al. designed a model with continuous
opinion values, where agents randomly interacted in pairs when the difference of their 
opinions was below a fixed threshold value. Ghaderi and Srikant studied the effect of
interactions among agents where an agent not only considers the opinions of his neighbors,
but also his inital opinion and how strongly he felt about it.






