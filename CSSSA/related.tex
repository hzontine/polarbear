
Gargiulo and Gandica\cite{gargiulo_role_2017} recently explored the connection
between homophily and polarization in the context of continuous-valued
opinions under a bounded confidence dynamic. The bounded confidence model
(originally in \cite{deffuant_mixing_2000,hegselmann_opinion_2002}) assumes
that agents will be influenced only by the opinions of others that their own
opinion is already sufficiently ``close" to (\textit{i.e.}, within some
threshold $\epsilon$); other opinions are viewed as too extreme from one's
own, and therefore untrustworthy. The result of such influence, when it does
occur, is an averaging operation that pulls each agent's opinion closer to the
other.

To build the initial graph for their simulation, Gargiulo and Gandica extend
the well-known preferential attachment mechanism
\cite{barabasi_emergence_1999} to incorporate homophily: when a new node
chooses which existing node to connect to in the graph, it incorporates
information not only about the degree of existing nodes, but also about the
similarity of their opinions to its own. In this way, not only are nodes with
more neighbors more likely to be chosen for attachment (as in
\cite{barabasi_emergence_1999}, \cite{deffuant_mixing_2000}), but nodes with
similar opinions to the new node are also more likely. Gargiulo's surprising
discovery was that as homophily is increased (\textit{i.e.}, as the second of
these two considerations is weighed more heavily), polarization -- measured as
the number of distinct opinion clusters at equilibrium -- \textit{decreases}.
This counterintuitive result can be explained as follows: when homophily is
low, a node does not form as many initial connections to nodes with similar
opinions (within the threshold $\epsilon$ of its own). Therefore, it is more
likely that that node will get ``stuck" during the bounded confidence process:
since it can't find many neighbors whose opinions seem plausible, it
stubbornly sticks to its own. Increasing homophily equips each node with more
neighbors whose opinions are close to its own, such that it can gradually be
nudged towards the emerging consensus.

Our work differs from Gargiulo's in that we use discrete opinions rather than
continuous; we implement a Binary Voter Model
process\cite{clifford_model_1973,holley_ergodic_1975} instead of Bounded
Confidence; and we measure polarization as the graph's \textit{assortativity}
rather than as the number of distinct opinion clusters. In this paper, we
explore whether Gargiulo's surprising results also hold under these different
assumptions.

