
\section{RELATED WORK}

Numerous models have been proposed to approximate the phenomenon of Opinion
Dynamics; we sketch only a few of the more important efforts here to give the
reader a feel for the breadth of the field.

Published independently by \cite{clifford_model_1973} and
\cite{holley_ergodic_1975} in the 1970s, the Binary Voter
Model (BVM) laid the initial foundation from which many other Opinion Dynamics
models have been constructed. The BVM, which represents an individual's
opinion as a single binary value, was intentionally simplified in several ways
in order for the sake of deriving an analytical solution. One simplification,
for instance, is that agents are distributed on a regular lattice rather than
an arbitrary graph. Periodically, a randomly chosen agent adopts the opinion
of a neighboring agent if the two opinions differ. Over time, the BVM will
always reach uniformity of opinion, according to Aldous and Fill.

\cite{yildiz_discrete_2011} expanded the BVM by adding a binary
``stubbornness" attribute to each agent. Stubborn agents never update their
opinion and 
% vvv ? reword?
others would often be changing depending whom they interacted with. They
discovered that the addition of only a few stubborn individuals always
resulted in a graph polarized by opinion.

Other classical models have been proposed where opinions are represented by
continuous values, typically between 0 and 1. \cite{weisbuch_interacting_2001}
designed a model with continuous opinion values, where agents only adopt the
opinion of a neighbor when the difference between their current opinions is
below a fixed threshold value. (This is termed ``bounded confidence.")
\cite{ghaderi_opinion_2012} further expanded this model by considering degrees
of stubbornness; agents differ in the degree to which they are biased in favor
of their initial opinion.

While many models assign each agent only one opinion value, some researchers
have explored models where agents have multiple opinion values.
\cite{deffuant_mixing_2000} gave each agent a vector of opinions, representing
opinions on different subjects. For the sake of simplification, they chose
binary values for these. Agents aggregate opinions through pairwise
evaluation, 
% vvv ? how can you "slightly shift" a binary value?
slightly shifting an agent's opinion after looping through all possible
pairings within the population. After more than 1000 interactions among the
agents, the researchers discovered orthogonalization of opinions, no
polarization, and no correlation between the opinion vectors.

% vvv I say we scrap this one unless we're going to say more
\cite{friedkin_social_1990} modeled an analytical approach which makes
simplifying assumptions about the process of opinion change. Latent vs.
expressed?

While the BVM restricts agents to only interact with their neighbors, other
models simulate interactions where agents consider the opinions of a group of
other agents in the graph. \cite{hegselmann_opinion_2002} (HK) constructed a
model where agents encounter others randomly from the whole population. The HK
model uses bounded confidence on a continuous opinion scale. Each randomly
selected agent aggregates all other agent opinions within his confidence bound
and considers the group average for shifting his opinion. In the DeGroot Model
\cite{degroot_reaching_1974}, an agent updates his opinion to be a weighted
average of his own opinion and the opinions of his neighbors.
 
*** Sirbu ***





