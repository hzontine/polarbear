
Opinion Dynamics (OD) models have a robust tradition, often traced to the
Binary Voter Model (BVM) of Holley and Liggett\cite{holley_ergodic_1975} and
Clifford and Sudbury\cite{clifford_model_1973}. OD models seek to reproduce
the phenomenon of individual agents forming opinions over time via mutual
influence, and to draw conclusions about the overall pattern of opinions that
may emerge in a society as a consequence of certain micro-behaviors.

Axelrod's work in this area\cite{axelrod_dissemination_1997} represented
agents with multiple discrete-valued attributes occupying a cellular grid.
Agents were more likely to influence one another when they had more attribute
values already in common, imitating what Axelrod called ``the fundamental
principle of human communication": that influence occurs more frequently
between people who perceive themselves as being already fairly alike. The
effect of influence in the model was to copy one of the differing attributes
from one agent to the other, further increasing their similarity. Among other
results, Axelrod demonstrated that as the \textit{range} of influence
increases (\textit{i.e.}, as agents are able to interact directly with other
agents 2, 3, 4, $\dots$ squares away), the degree of overall homogeneity in
the society increases. He measured homogeneity as the number of distinct
opinion clusters that emerge in the population.

Axelrod's result might lead us to predict that polarization would
\textit{decrease} with accessibility, rather than increase. Shibani \textit{et
al.}\cite{shibanai_effects_2001} and Grieg\cite{greig_end_2002} found that
subtle changes to the model, however, can produce the opposite result. Too,
Flache and Macy\cite{flache_why_2006} concluded that Axelrod's original result
depended crucially on the opinions being discrete valued; when continuous
opinions were used, and adjustments could be made to them gradually,
polarization actually increased with range of influence.

One popular approach to modeling \textit{continuous} opinion dynamics is the
Bounded Confidence (BC) model (originally in
\cite{deffuant_mixing_2000,hegselmann_opinion_2002}). This assumes that agents
will be influenced only by the opinions of others that their own opinion is
already sufficiently ``close" to (\textit{i.e.}, within some threshold
$\epsilon$); other opinions are viewed as too extreme from one's own, and
therefore untrustworthy. (This is similar in spirit to Axelrod's ``fundamental
principle," but in the context of a single attribute, not an array of them.)
The result of such influence, when it does occur, is an averaging operation
that pulls each agent's opinion closer to the other. The BC mechanism is one
way of preventing a graph of continuous opinions from converging to absolute
homogeneity, as will happen if the averaging operation happens
unconditionally.

All of these studies inspired by Axelrod place agents on a rectangular grid.
Work has been done, too, on agents connected via a more general network/graph
structure, whether a complete graph, possibly with edge
weights\cite{deffuant_how_2002,degroot_reaching_1974}, or a general
graph\cite{dandekar_biased_2013}. Several of these studies have explored the
interplay between homophily and (various definitions of) polarization in the
context of continuous-valued opinions. Dandekar \textit{et
al.}\cite{dandekar_biased_2013} in particular use a variant of the
assortativity coefficient called the network degree index (for continuous
attributes). However, their work was still based on continuous attributes, and
they invoke a more complex opinion formation process than we do, incorporating
confirmation bias.

Recently, Gargiulo and Gandica\cite{gargiulo_role_2017} explored the
connection between homophily and polarization in the context of
continuous-valued opinions under a BC dynamic. To build the initial graph for
their simulation, they extend the well-known preferential attachment mechanism
\cite{barabasi_emergence_1999} to incorporate homophily: when a new node
chooses which existing node to connect to in the graph, it incorporates
information not only about the degree of existing nodes, but also about
the similarity of their opinions to its own. In this way, not only are
nodes with more neighbors more likely to be chosen for attachment (as in
\cite{barabasi_emergence_1999}, \cite{deffuant_mixing_2000}), but nodes with
similar opinions to the new node are also more likely. 

Gargiulo and Gandica discovered that under these assumptions, as homophily is
increased (\textit{i.e.}, as similarity is weighed more heavily than degree
when attaching new nodes), polarization -- measured as the number of distinct
opinion clusters at equilibrium -- \textit{decreases}. This counterintuitive
result can be explained as follows: when homophily is low, a node does not
form as many initial connections to nodes with similar opinions (within the
threshold $\epsilon$ of its own). Therefore, it is more likely that that node
will get ``stuck" during the bounded confidence process: since it can't find
many neighbors whose opinions seem plausible, it stubbornly sticks to its own.
Increasing homophily equips each node with more neighbors whose opinions are
close to its own, such that it can gradually be nudged towards the emerging
consensus.

Our work differs from these other studies in that we use a graph instead of a
cellular grid; discrete opinions rather than continuous; we implement a BVM
process instead of BC; and most importantly, we measure polarization as the
graph's \textit{assortativity} rather than as the number of distinct opinion
clusters or the extremity of views.
