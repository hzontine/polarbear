
\section{RELATED WORK}

Opinion Dynamics is the study of how individuals formulate and change their personal 
opinions based on interactions with others. Published independently by Clifford and 
Holley in the 1970s, the Binary Voter Model (BVM) laid the inital foundation from 
which other models were constructed from. The BVM, which represents opinions as a 
single binary value, was intentionally simplified for the sake of an analytical 
solution. Periodically, a randomly chosen agent adopts the opinion of a directly 
connected agent if the two opinions differ. Over time, the BVM will always reach 
uniformity of opinion, according to Aldous and Fill. Yildiz et al. redesigned the
BVM by adding binary stubbornness attribute to each agent. Some would never update
their opinion and others would often be changing depending whom they interacted with.
They discovered that the addition of only a few stubborn individuals always resulted 
in a graph polarized by opinion.

Other classical models have been proposed where opinions are represented by continuous 
values, typically between 0 and 1. Weisbuch et al. designed a model with continuous
opinion values, where agents randomly interacted in pairs when the difference of their 
opinions was below a fixed threshold value. Ghaderi and Srikant studied the effect of
interactions among agents where an agent not only considers the opinions of his 
neighbors, but also his inital opinion and how strongly he felt about it.

While many models assign each agent only one opinion value, some researchers explore 
models where agents have multiple opinion values. Deffuant et al. 2000 modeled agent's
opinions as a vector of opinions, representing opinions on different subjects. For the
sake of simplication, they chose binary values. Agents aggregate opinions through
pairwise evaulation, slightly shifting an agent's opinion after looping through all 
possible pairings within the population. After more than 1000 interactions among
the agents, the researchers discovered orthogonalisation of opinions, no polarization, 
and no correlation between the opinion vectors.

Friedkin and Johnsen 1990 modeled an analytical approach which makes simplifying 
assumptions about the process of opinion change. Latent vs. expressed?

While the BVM restricts agents to only interact with their neighbors, other models
simulate interactions where agents consider the opinions of a group of other agents 
in the graph. Hegselmann et al. (HK) constructed a model where agents encounter others
randomly from the whole population. The HK model uses bounded confidence on a continuous
opinion scale. Each randomly selected agent aggregates all other agent opinions within
his confidence bound and considers the group average for shifting his opinion. In the 
DeGroot Model, an agent updates his opinion to be a weighted average of his own opinion 
and the opinions of his neighbors.
 
*** Sirbu ***










