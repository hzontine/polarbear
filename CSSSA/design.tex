
\section{EXPERIMENTAL DESIGN}

We implemented a factorial design with the above two factors:
``\textsl{selection method}" (with levels of \textbf{with replacement} and
\textbf{without replacement}) and ``\textsl{influence direction}" (with levels
of \textbf{neighbor influences node} and \textbf{node influences neighbor}).
Our response variable was the \textsl{time to convergence}; \textit{i.e.}, the
number of iterations (pairwise encounters between nodes) before total
uniformity of opinion was reached.

We ran each of the four combinations of factor levels for 200 trials. Each
trial began with an Erdos-Renyi random graph \cite{erdos_random_1959} with 100
nodes and an edge probability of .04, using the R \texttt{igraph} package
\cite{igraph}. If the random graph generated for a given trial turned out not
to be \textit{connected} (\textit{i.e.}, not all nodes were reachable from all
others via some walk), that graph was discarded and another generated until a
connected graph was produced. (This is because the model does not guarantee
convergence to uniformity of opinion for unconnected graphs.)

Each node was initially assigned one of the two opinion values at random, with
a probability of p=.5 for each opinion value. 

All of the initial graphs were generated once and then re-used for each of the
four treatment combinations, to ensure that every treatment combination was
simulated with the identical 200 initial conditions.

