
\section{Background}

The majority of poll results published before the Presidential Election
predicted Donald Trump's defeat 'by a landslide' \citep{pomarico_nate_2016}.
On Election Day, in spite of this, he secured more than enough electoral votes
to win him the presidency. What caused poll results to differ so significantly
from election results? The sole purpose of polls is to represent the voting
intentions of the population. Polling every eligible voter would require
unfathomable resources; as a result, all polls utilize only a sample of the
population. Unless pollsters fail to accurately sample from the voting
population, their results should be similar to the votes each candidate
receives. This issue came to light when Franklin D. Roosevelt beat Alfred
Landon in the 1936 Presidential election. Pollsters realized that by
conducting the poll explicitly over the phone, they were choosing to sample
from voters who could afford a house phone; they neglected to poll from an
accurate sample of the voting population. Since that election, pollsters have
aimed to not repeat that mistake. So what if there is a different effect at
work that caused poll results to significantly differ from election results?
What if people told pollsters their intentions were to vote against Trump,
but, on election day, voted for him?

