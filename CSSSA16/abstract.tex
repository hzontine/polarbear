
\section*{ABSTRACT}

Opinion Dynamics models seek to reproduce the phenomenon of individual agents
forming opinions over time via mutually influencing one another. As with all
agent-based models (ABMs), researchers seek to faithfully model the real-world
process through a selective simplification of the phenomenon.

When creating a simulation to realize such a model, various implementation
choices present themselves that at first glance may look arbitrary. In some
cases, however, the way in which the details of the simulation are implemented
can lead to important differences in its overall behavior; in the extreme,
incorrect conclusions may even be drawn about what macro-behavior the abstract
model is guaranteed to produce. In this paper, we look at two such choices
germane to Opinion Dynamics models: a subtle difference in the way agents are
randomly chosen for interaction, and the direction of influence between the
two agents in such an encounter. In both cases, the rate of convergence to
equilibrium is profoundly affected by what seem to be arbitrary implementation
choices. This raises questions about the robustness of certain conclusions
that are sometimes drawn from ABMs, and suggests strategies for avoiding
pitfalls.
